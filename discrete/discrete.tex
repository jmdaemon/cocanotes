\documentclass[11pt]{article}

% Imports
\usepackage[table]{xcolor}
\usepackage[
    noparindent,
    usearialheadings,
    usenewpagebeforesection,
    ]{coco}
\usepackage[
    numbersets,
    logic,
    ]{shortmath}
\usepackage{code}
\usepackage{discrete}
\usepackage{thmstyles}
\usepackage{boxstyles}
\usepackage{changepage}

\usepackage{wrapfig}
\usepackage{setspace}
\usepackage{makecell}
\usepackage{adjustbox}
\usepackage{titlesec}

% Make better use of the page
% TODO: We'll keep margins like this until we add in margin notes
\geometry{
    margin=2cm,
}

\setlength{\lineskip}{0.0pt}
\setlength{\lineskiplimit}{0pt}
\setlength{\baselineskip}{0pt}

\newcommand*\qistmt[4]{#1 $x \in #2$, if #3 then #4}
\newcommand*\qstmt[3]{#1 $x \in #2$ such that #3}

%\newenvironment\stmt{}
%\newcommand\h1[1] {
%\section*{#1}
%}
\newcommand\inv{\ensuremath{^{-1}}}

\begin{document}

% Main cover page
\begin{titlepage}
    \begin{center}
        \vspace*{1cm}
            
        \Huge
        \textbf{Discrete Mathematics}
            
        \vspace{0.5cm}
        \LARGE
        Discrete Mathematics for Computer Science
            
        \vspace{1.5cm}
            
        \textbf{Joseph M. R. Diza}
            
        \vfill
            
    \end{center}
\end{titlepage}


% TODO: Add fancyhdr shit here

% Table of Contents
\tableofcontents

% Quick Introduction
\section{Introduction}

This document contains various definitions, laws, notation for Discrete Mathematics.

The content is stripped of many examples for brevity purposes, and is essentially a reference book.

% Content
\section{Logical Statements}

\begin{definition}[Propositions]\label{def:props}
Declarative true or false (but not both) statements.
\end{definition}

\begin{definition}[Primitive Statements]\label{def:prims}
Propositions that are in their simplest terms.
\end{definition}

\begin{definition}[Universal Statements]\label{def:universal-stmts}
Statement that a property holds true for all elements in a set.
\end{definition}

\begin{definition}[Conditional Statements]\label{def:cond-stmts}
Statement that if a condition is true then another condition is true.
\end{definition}

\begin{definition}[Existential Statements]\label{def:exist-stmts}
Statement that is true for at least one element in a set, given a property is a statement.
\end{definition}

% TODO:
% Fill in the rest of the missing sections:

\section{Set Theory}

\begin{definition}[Set]\label{def:set}
    A set is an arbitrary collection of elements.
\end{definition}

\subsection{Notation}

\subsubsection{Set Roster Notation}

Set-Builder notation (also known as Set Roster Notation) is used to describe sets.

\begin{table}[!htbp]
    \centering
    \caption{Set Roster Notation}
		\rowcolors{2}{}{gray!10}
    \begin{tabular}{ c l l l }
        \toprule
        Symbol  & Name          & Meaning           & Usage \\
        \midrule
        \{ \}   & Curly Braces  & The set of all    & Denotes a set \\
        |       & Pipe/Bar      & Such that         & Indicates conditions on sets \\
        ...     & Ellipses      & And so forth      & Elides numbers from the entire set for brevity \\
        \bottomrule
    \end{tabular}
    \label{tab:tbl-set-notation}
\end{table}

\subsubsection{Set Theory Notation}

\begin{table}[!htbp]
    \centering
    \caption{Set Theory Notation}
    \rowcolors{2}{}{gray!10}
    \begin{tabular}{ c l l l }
        \toprule
        Symbol      & LaTeX             & Name              & Meaning \\
        \midrule
        $\in$       & \verb|\in|        & In                & Is an element of ... \\
        $\notin$    & \verb|\notin|     & Not in            & Is not an element of ... \\
        $\subset$   & \verb|\subset|    & Proper Subset     & Is a proper subset of ... \\
        $\subseteq$ & \verb|\subseteq|  & Subset            & Is a subset of ... \\
        $\subset$   & \verb|\subset|    & Proper Superset   & Is a proper superset of ... \\
        $\subseteq$ & \verb|\subseteq|  & Superset          & Is a superset of ... \\
        $\cup$      & \verb|\cup|       & Set Union         & The set of elements in either set A or B \\
        $\cap$      & \verb|\cap|       & Set Intersection  & The set of elements in both set A and B \\
        $\setminus$ & \verb|\setminus|  & Set Difference    & The set of elements of A that are not in B \\
        \bottomrule
    \end{tabular}
    \label{tab:tbl-set-theory-notation}
\end{table}

\subsubsection{Interval Notation}
\begin{definition}[Interval Notation]\label{def:interval-notation}
    Given real numbers $a$ and $b$ with $a \leq b$
    \begin{align*}
        (a, b) = \{ x \in \R \mid a < x < b \} \\
        (a, b] = \{ x \in \R \mid a < x \leq b \} \\
        [a, b) = \{ x \in \R \mid a \leq x < b \} \\
        [a, b] = \{ x \in \R \mid a \leq x \leq b \}
    \end{align*}

    For unbounded intervals at infinity:

    \begin{align*}
        (a, \infty) = \{ x \in \R \mid x > a \} \\
        [a, \infty) = \{ x \in \R \mid x \geq a \} \\
        (-\infty, b) = \{ x \in \R \mid x < a \} \\
        (-\infty, b] = \{ x \in \R \mid x \leq a \}
    \end{align*}
\end{definition}

\newpage

\subsection{Order}

\subsubsection{Tuples}

\begin{definition}[Ordered Pair]\label{def:ordered-pair}
    A pair of elements $(a, b)$ where a is the first
    element of the pair, and b is the second. 
\end{definition}

\begin{definition}[Ordered Pair Equality]\label{def:ordered-pair-eq}
    Two ordered pairs $(a,b)$, $(c,d)$ are equal if and only if
    $a = c$ and $b = d$. In other words:

    \begin{equation}
        (a,b) = (c,d) \iff (a = c) \land (b = d)
    \end{equation}
\end{definition}

\begin{definition}[Tuples]\label{def:tuple}
    Let $n$ be a positive integer and let $x_1, x_2, ... , x_n$ be elements.

    An ordered $n$-tuple $(x_1, x_2, ... , x_n)$ is sorted in ascending order
    starting with $x_1$ to $x_n$. An ordered 2-tuple is an ordered pair,
    and an ordered 3-tuple is an ordered triple.
\end{definition}

\begin{definition}[Tuple Equality]\label{def:tuple-eq}
    Two ordered n-tuples $(x_1, x_2, ..., x_n)$, $(y_1, y_2, ..., y_n)$
    are equal if and only if:

    \begin{equation}
        (x_1, x_2, ..., x_n) = (y_1, y_2, ..., y_n) \iff x_1 = y_1, ..., x_n = y_n
    \end{equation}
\end{definition}

In other words, all elements of one tuple must be equal to all the elements of the other tuple.

\subsubsection{Cartesian Product}

\begin{definition}[Cartesian Product]\label{def:cartesian-product}
    Given sets $A_1, A_2, ..., A_n$, the Cartesian product of $A_1, A_2, ..., A_n$
    denoted

    $A_1 \times A_2 \times ... \times A_n$ is the set of all ordered $n$-tuples
    $(a_1, a_2, ..., a_n)$ where $a_1 \in A_1, a_2 \in A_2, ... a_n \in A_n$
    \begin{equation}
        \begin{aligned}
            A_1 \times A_2 \times ... \times A_n &= \{ (a_1, a_2, ..., a_n) | a_1 \in A_1, a_2 \in A_2, ... a_n \in A_n \} \\
            A_1 \times A_2 &= \{ (a_1, a_2) | a_1 \in A_1, a_2 \in A_2 \}
        \end{aligned}
    \end{equation}
\end{definition}

\subsubsection{Strings}
\begin{definition}[Strings]\label{def:string}
    Let $n$ be a positive integer. Given a finite set $A$, a string
    of length $\frac{n}{A}$ is an ordered $n$-tuple of elements of $A$.
    The elements of $A$ are called characters. The null string of $A$
    denoted $\lambda$ is a string with no characters and length of $0$.
    If $A = \{0, 1\}$ then a string over $A$ is a bit string.
\end{definition}

\subsection{Set Types}

\subsubsection{Subset}

\begin{definition}[Subset]\label{def:subset}
    If A, B are sets, then A is a subset of B if and only if every element of A is an element of B.

    Symbolically:
    \begin{align*}
        A \subseteq B &\iff \forall x, \text{ if $x \in A$, then $x \in B$} \\
        \text{ The negation is therefore: } & \\
        A \not \subseteq B &\iff \exists x \text{ such that $x \in A$ and $x \notin B$}
    \end{align*}

    \starON
    **Note:** Contained-in is another way of saying 'is a subset of'.
    \starOFF
\end{definition}

\subsubsection{Proper Subset}

\begin{definition}[Proper Subset]\label{def:proper-subset}
    Let $A$, $B$ be sets, $A$ is a proper subset of $B$ if and only if every element of $A$ is in $B$, but there
    is at least one element of $B$ not in $A$.

    In other words. $A$ is a proper subset of $B$ $\iff$
    \begin{enumerate}
        \item $A \subseteq B$ and
        \item there is at least one element in $B$ that is not in $A$
    \end{enumerate}
\end{definition}

\subsubsection{Empty Set}

\begin{definition}[Empty Set]\label{def:empty-set}
    The empty set (or null set) is the set that contains no elements.
\end{definition}

\begin{theorem}[The Empty Set is a Subset of Every Set]\label{thm:empty-set}\bl
    If $E$ is a set with no elements and $A$ is any set, then $E \subseteq A$.
\end{theorem}

\begin{corollary}[Empty Set is Unique]\label{cor:empty-set-unique}\bl
    There is only one set with no elements.
\end{corollary}

% Disjoint Sets
\subsubsection{Partitions of Sets}

\subsubsection{Disjoint Sets}
\begin{definition}[Disjoint]\label{def:disjoint}
    Two sets are disjoint if, and only if they have no elements in common.
    \begin{equation*}
        \text{$A$ and $B$ are disjoint } \iff A \cap B = \O
    \end{equation*}
\end{definition}


\subsubsection{Mutually Disjoint Sets}
\begin{definition}[Mutually Disjoint]\label{def:mutually-disjoint}
    Sets $A_1$, $A_2$, $A_3$, ..., are mutually disjoint (nonoverlapping)
    if, and only if, no two sets $A_i$, $A_j$ with distinct subscripts
    in common.
    \begin{equation*}
        A_i \cap A_j = \O
    \end{equation*}
    Where $i$, $j$ are two integers such that $i \neq j$.
\end{definition}

\begin{definition}[Partition]\label{def:partition}
    A finite or infinite collection of non-empty sets $\{ A_1, A_2, A_3, ... \}$
    is the partition of a set $A$, if, and only if,
    \begin{enumerate}
        \item $A$ is the union of all $A_i$ and
        \item the sets $\{ A_1, A_2, A_3, ... \}$ are mutually disjoint.
    \end{enumerate}
\end{definition}

\subsubsection{Power Set}

\begin{definition}[Power Set]\label{def:power-set}
    Given a set $A$, the power set of $A$, denoted $\powerset$ is the set of all subsets of $A$.
\end{definition}

% Venn Diagrams


\newpage
\subsection{Properties of Sets}

\subsubsection{Subset Relations}

\begin{definition}[Subset Relations]\label{def:subset-relations}
    Note, keep in mind that union, intersection and difference have higher precedence
    than set inclusion. E.g $A \cap B \subseteq C$ means $(A \cap B) \subseteq C$.

    \begin{enumerate}
        \item Inclusion of Intersection: For all sets $A$ and $B$,
            \begin{equation*}
                \text{(a) } A \cap B \subseteq A \text{ and (b) } A \cap B \subseteq B
            \end{equation*}
        \item Inclusion in Union. For all sets $A$ and $B$
            \begin{equation*}
                \text{(a) } A \subseteq A \cup B \text{ and (b) } B \subseteq A \cup B
            \end{equation*}
        \item Transitive Property of Subsets: For all sets $A$, $B$ and $C$
            \begin{equation*}
                \text{if } A \subseteq B \text{ and } B \subseteq C, \text{ then } A \subseteq C
            \end{equation*}
    \end{enumerate}
\end{definition}

\begin{definition}[Procedural Set Definitions]\label{def:procedural-set-defs}
    Let $X$ and $Y$ be subsets of a universal set $U$ and supposed $x$ and $y$ are elements of $U$.
    \begin{enumerate}
        \item $x \in X \cup Y \iff x \in X \text{ or } x \in Y$
        \item $x \in X \cap Y \iff x \in X \text{ and } x \in Y$
        \item $x \in X - Y \iff x \in X \text{ and } x \notin Y$
        \item $x \in X^c \iff x \notin X$
        \item $(x,y) \in X \times Y \iff x \in X \text{ and } y \in Y$
    \end{enumerate}
\end{definition}

\newpage
\subsubsection{Set Identities}

\begin{definition}[Identity]\label{def:identity}
    An identity is an equation that is universally true for all elements of a set.
\end{definition}

Let all sets referred to below be subsets of a universal set $U$.

\begin{table}[!htbp]
        \caption{Set Identities}
        \centering
        \rowcolors{2}{}{gray!10}
        \begin{adjustwidth}{-1.5cm}{-1.5cm}
        \centering
        \adjustbox{width=1.1\textwidth}{
        \begin{tabular}{ *{4}{l} }
            \toprule
            Set Identity & Quantifier & Form & Alternative Form \\
            \midrule
            Commutative Law         & For all sets $A$ and $B$
                                    & $A \cup B = B \cup A$ 
                                    & $A \cap B = B \cap A$ \\
            Associative Law         & For all sets $A$, $B$, and $C$
                                    & $(A \cup B) \cup C = A \cup (B \cup C)$
                                    & $(A \cap B) \cap C = A \cap (B \cap C)$ \\
            Distributive Law        & For all sets $A$, $B$, and $C$
                                    & $A \cup (B \cap C) = (A \cup B) \cap (A \cup C)$
                                    & $A \cap (B \cup C) = (A \cap B) \cup (A \cap C)$ \\
            Identity Law            & For every set $A$
                                    & $A \cup \O = A$ 
                                    & $A \cap U = A$ \\
            Complement Law          & For every set $A$
                                    & $A \cup A^c = U$
                                    & $A \cap A^c = \O$ \\
            Double Complement Law   & For every set $A$
                                    & $(A^c)^c = A$ 
                                    & \\
            Idempotent Law          & For every set $A$
                                    & $A \cup A = A$
                                    & $A \cap A = A$ \\
            Universal Bound Law     & For every set $A$
                                    & $A \cup U = U$
                                    & $A \cap \O = \O$ \\
            DeMorgan's Law          & For all sets $A$, and $B$
                                    & $(A \cup B)^c = A^c \cap B^c$
                                    & $(A \cap B)^c = A^c \cup B^c$ \\
            Absorption Law          & For all sets $A$, and $B$
                                    & $A \cup (A \cap B) = A$
                                    & $A \cap (A \cup B) = A$ \\
            Complements of
                $U$ and $\O$        &
                                    & $U^c = \O$
                                    & $\O^c = U$ \\
                Set Difference Law  & For all sets $A$, and $B$
                                    & $A - B = A \cap B^c$
                                    & \\
            \bottomrule
        \end{tabular}
        \label{tab:tbl-set-identities}
        }
        \end{adjustwidth}
    \end{table}

\begin{definition}[Prove Set Equality]\label{def:prove-set-eq}
    Let sets $X$ and $Y$ be given. To prove $X = Y$
    \begin{enumerate}
        \item Prove that $X \subseteq Y$
        \item Prove that $Y \subseteq X$
    \end{enumerate}
\end{definition}

% Theorem
\begin{definition}[Intersection and Union with a Subset]\label{def:intersect-union-subset}
    For any sets $A$ and $B$, if $A \subseteq B$, then
    \begin{equation*}
        \text{(a) } A \cap B = A \text{ (and) } A \cup B = B
    \end{equation*}
\end{definition}

\section{Predicate Logic}

\subsection{Logical Operators}
\begin{table}[!htbp]
    % Logical Symbol    , Programming Symbol    , LaTeX Symbol  , Name          , Meaning
    % &&                , &&                    , \wedge        , Logical And   , Conjunction of two statements
    % ||                , ||                    , \lor          , Logical Or    , Disjunction of two statements
    % Not               , !                     , \lnot         , Negation      , Negates a statement
    % Xor               , xor                   , \xor          , Exclusive Or  , Logical Or, but both statements cannot be true
    \centering
    \caption{Logical Operators}
		\rowcolors{2}{}{gray!10}
    \begin{tabular}{ c c l l l l  }
        \toprule
        Symbol      & Programming   & LaTeX         & Name          & Meaning \\
        \midrule
        $\land$     & \&\&          & \verb|\wedge| & And           & Conjunction of two terms \\
        $\lor$      & ||            & \verb|\lor|   & Or            & Disjunction of two terms \\
        $\lxor$     & \string^      & \verb|\lxor|  & Xor           & Exclusive logical or \\
        $\neg$      & !             & \verb|\neg|   & Not           & Negation of a term \\
        $\to$       &               & \verb|\to|    & Implies       & Logical implication \\
        $\iff$      &               & \verb|\iff|   & Biconditional & Logical Equivalence \\
        \bottomrule
    \end{tabular}
    \label{tab:tbl-logic-ops}
\end{table}

\starON
**Note:** An alternative syntax for negation is tilde: $\sim$ (\verb|\sim|).

**Note:** An alternative syntax for logical equivalence is $\equiv$ (\verb|\equiv|).

**Note:** In the biconditional, the hypothesis implies the conclusion and vice versa.

**Note:** Exclusive or contains almost the same truth values as logical or, with the exception
that both operands of the connective cannot both be simultaneously true.
\starOFF

\subsubsection{Truth Values}

$T$ is used to denote a true value and $F$ is used to denote a false value.

\begin{definition}[Negation]\label{def:negation}
    The interchanging of truth values in a statement.
    Given statement p, the resulting statement is "not p"
\end{definition}

\begin{definition}[Conjunction]\label{def:conjunction}
    A compound statement read as "p and q" $p \land q$ that 
    is true only when both p and q are true, and false otherwise.
\end{definition}

\begin{definition}[Disjunction]\label{def:disjunction}
    A compound statement read as "p or q" $p \lor q$
    that is false when both p and q are false, and true otherwise.
\end{definition}

\begin{definition}[Statement Form]\label{def:statement-form}
    An expression of statement variables
    and logical connectives that becomes a proper statement
    when actual statements are substituted in for the variables.
\end{definition}

\begin{definition}[Truth Table]\label{def:truth-table}
    The enumeration of all possible truth values for a given statement.
\end{definition}

\begin{definition}[Logical Equivalence]\label{def:logical-eq}
    Two statements are logically equivalent (denoted $\iff$),
    if and only if they have identical truth values
    for every possible substitution for the statement variables. 
\end{definition}

\large{Testing Statements $P$, $Q$ for Logical Equivalence}
\begin{enumerate}
    \item Create the truth tables for $P$, $Q$
    \item For all truth values, check that $P$'s values match $Q$'s
\end{enumerate}

If for every row of $P$, the truth value is the same as $Q$, then 
the two forms are logically equivalent. Otherwise if even one
value is different, the two are not.

\begin{definition}[Tautology]\label{def:tautology}
A statement form that holds true for all possible statement substitutions regardless of the individual statements truth values.
\end{definition}

\begin{definition}[Contradiction]\label{def:contradiction}
A statement form that holds false for all possible statement substitutions regardless of the individual statements truth values.
\end{definition}

\starON
**Note:** $T$ or $t$ denotes a tautology

**Note:** $F$ or $c$ denotes a contradiction

The usage depends on whichever variant is used. In our case we will use
$t$ and $c$.
\starOFF

\subsection{Laws of Logic}

\begin{table}[!htbp]
    \centering
    \rowcolors{2}{}{gray!10}
    \begin{tabular}{ l l l }
        \toprule
        Logical Law & Form & Additional Form \\
        \midrule
        Commutative Law     & $p \lor q \iff q \lor p$
                            & $p \land q \iff q \land p$ \\
        \addlinespace[0.5em]
        Associative Law     & $p \lor (q \lor r) \iff (p \lor q) \lor r$
                            & $p \land (q \land r) \iff (p \land q) \land r$ \\
        \addlinespace[0.5em]
        Distributive Law    & $p \lor (q \land r) \iff (p \lor q) \land (p \lor r)$ 
                            & $p \land (q \lor r) \iff (p \land q) \lor (p \land r)$ \\
        \addlinespace[0.5em]
        Inverse Law         & $p \lor \neg p \iff t$
                            & $p \land \neg p \iff c$ \\
        \addlinespace[0.5em]
        Identity Law        & $p \lor c \iff p$
                            & $p \land t \iff p$ \\
        \addlinespace[0.5em]
        Double Negation Law & $\neg \neg p \iff p$ & \\
        \addlinespace[0.5em]
        Idempotent Law      & $p \lor p \iff p$ 
                            & $p \land p \iff p$ \\
        \addlinespace[0.5em]
        Domination Law      & $p \lor t \iff t$
                            & $p \land c \iff c$ \\
        \addlinespace[0.5em]
        DeMorgan's Laws     & $\neg(p \lor q) \iff \neg p \land \neg q$
                            & $\neg(p \land q) \iff \neg p \lor \neg q$ \\
        \addlinespace[0.5em]
        Absorption Laws     & $p \lor (p \land q) \iff p$
                            & $p \land (p \lor q) \iff p$ \\
        \addlinespace[0.5em]
        \bottomrule
    \end{tabular}
    \label{tab:tbl-laws-of-logic}
    \caption{Laws of Logic. These statements hold true for any statements $p$, $q$, $r$, any tautology $t$, and contradiction $c$.}
\end{table}

\starON
**Note:** The Inverse Law is also known as the Negation Law.

**Note:** The Domination Law is also known as the Universal Bound Law.
\starOFF

\subsection{Conditional Statements}

% TODO: Truth table for p -> q
\begin{definition}[Conditional]\label{def:conditional}
    If $p$, $q$ are statement variables, the conditional of $q$ by $p$
    is a statement that is false when $p$ is true, and $q$ is false, and true
    otherwise. $p$ is known as the hypothesis and $q$ is the conclusion.
\end{definition}

Conditional statements (also known as implications) are read as
"if $p$ then $q$" or "$p$ implies $q$".

Conditional statements are vacuously true (true by default).

\subsection{Variants of Implication}

\begin{definition}[Contrapositive]\label{def:contrapositive}
    The contrapositive of conditional statement: "if $p$ then $q$",
    is "if $\neg q$ then $\neg p$". Symbolically:
    \begin{equation}
        p \to q \iff \neg q \to \neg p
    \end{equation}
\end{definition}

\begin{definition}[Converse]\label{def:converse}
    Given conditional statement: "if $p$ then $q$"
    The converse is "if $q$ then $p$":
\end{definition}

\begin{table}[!htbp]
    \centering
    \rowcolors{2}{}{gray!10}
    \begin{tabular}{ l c c}
        \toprule
        Conditional Statement & Form                & $\iff$ $p \to q$ \\
        \midrule
        Implication         & $p \to q$             & Yes \\
        Contrapositive      & $\neg q \to \neg p$   & Yes \\
        Converse            & $q \to p$             & No \\
        Inverse             & $\neg p \to \neg q$   & No \\
        \bottomrule
    \end{tabular}
    \label{tab:tbl-implication-variants}
    \caption{Variants of Conditional Statements}
\end{table}

\starON
**Note:** Although the converse and inverse are not logically equivalent to "if $p$ then $q$"
they are logically equivalent to each other.
\starOFF

\begin{definition}[Only If]\label{def:only-if}
    For statements $p$, $q$,
    $p$ only if $q$ means "if not $q$ then not $p$".
    or equivalently "if $p$ then $q$".
\end{definition}

\begin{definition}[Biconditional]\label{def:biconditional}
    The biconditional of statements $p$, $q$ is
    "$p$ if and only if $q$" denoted $p \iff q$. It
    is true if both $p$ and $q$ share the same truth values
    and false otherwise.
\end{definition}

\subsection{Logical Order of Operations}
\begin{enumerate}
    \item $\neg$
    \item $\land$, $\lor$ (Parenthesis may be needed).
    \item $\to$, $\iff$ (Parenthesis may be needed).
\end{enumerate}

\begin{definition}[Sufficient Condition]\label{def:sufficient-condition}
    For statements $r$, $s$,
    $r$ is a sufficient condition for $s$ implies "if $r$ then $s$".
\end{definition}

\begin{definition}[Necessary Condition]\label{def:necessary-condition}
    For statements $r$, $s$,
    $r$ is a necessary condition for $s$ implies "if not $r$ then not $s$".
\end{definition}

\subsection{Arguments}

% TODO: Arguments definition: Page: 66

\subsection{Rules of Inference}
\begin{table}
    \caption{Rules of Inference}
    \begin{adjustwidth}{-1.5cm}{-1.5cm}
    \centering
    \adjustbox{width=1.1\textwidth}{
    \begin{tabular}[f]{ *{4}{cc} }
        \toprule
        Inference Rule & Argument Form & Alternate Form  & Inference Rule & Argument Form & Alternate Form \\
        \midrule
        Modus Ponens    &
                        \begin{argument}
                            \premise{p \to q}
                            \premise{p}
                            \conclusion{q}
                        \end{argument}
                        & &
        Elimination     & 
                        \begin{argument}
                            \premise{p \lor q}
                            \premise{\neg q}
                            \conclusion{p}
                        \end{argument}
                        &
                        \begin{argument}
                            \premise{p \lor q}
                            \premise{\neg p}
                            \conclusion{q}
                        \end{argument} \\
        \midrule
        Modus Tollens   &
                        \begin{argument}
                            \premise{p \to q}
                            \premise{\neg q}
                            \conclusion{\neg p}
                        \end{argument}
                        & &
        Transitivity    &
                        \begin{argument}
                            \premise{p \to q}
                            \premise{q \to r}
                            \conclusion{p \to r}
                        \end{argument}
                        & \\
        \midrule
        Generalization  &
                        \begin{argument}
                            \premise{p}
                            \conclusion{p \lor q}
                        \end{argument}
                        &
                        \begin{argument}
                            \premise{q}
                            \conclusion{p \lor q}
                        \end{argument}
                        &
        Proof By
        Division into Cases
                        &
                        \begin{argument}
                            \premise{p \lor q}
                            \premise{p \to r}
                            \premise{q \to r}
                            \conclusion{r}
                        \end{argument}
                        & \\
        \midrule
        Specialization  &
                        \begin{argument}
                            \premise{p \land q}
                            \conclusion{p}
                        \end{argument}
                        &
                        \begin{argument}
                            \premise{p \land q}
                            \conclusion{q}
                        \end{argument}
                        &
        Contradiction
        Rule
                        &
                        \begin{argument}
                            \premise{\neg p \to c}
                            \conclusion{p}
                        \end{argument}
                        & \\
        \midrule
        Conjunction     &
                        \begin{argument}
                            \premise{p}
                            \premise{q}
                            \conclusion{p \land q}
                        \end{argument}
                        & \\
        \bottomrule
    \end{tabular}
    }
    \end{adjustwidth}
    \label{tab:tbl-inference-rules}
\end{table}

% TODO: Entire section of gotchas

\subsection{Quantifiers}

\begin{table}[!htbp]
    % Table of quantifiers
    % Symbol    , Name          , Meaning
    % ∀         , Universal     , For all elements of the set
    % E         , Existential   , There exists an element of the set
    \centering
    \begin{tabular}{ c c l l  }
        \toprule
        Symbol      & LaTeX             & Name                      & Meaning \\
        \midrule
        $\exists$   & \verb|\exists|    & Existential Quantifier    & There exists at least one \\
        $\forall$   & \verb|\forall|    & Universal Quantifier      & For all \\
        \bottomrule
    \end{tabular}
    \label{tab:tbl-quantifiers}
    \caption{Quantifiers}
\end{table}

\begin{table}[!htbp]
    \centering
    \begin{tabular}{ c l }
        \toprule
        Name & Rule \\
        \midrule
        Universal Instantiation &
        \begin{argument}
            \premise{\forall x P(x)}
            \conclusion{P(c)}
        \end{argument} \\
        \addlinespace[0.5em]

        Universal Generalization &
        \begin{argument}
            \premise{P(c)}
            \conclusion{\forall x P(x)}
        \end{argument} \\
        \addlinespace[0.5em]

        Existential Instantiation &
        \begin{argument}
            \premise{\exists x P(x)}
            \conclusion{P(c) \text{ for some element of } c}
        \end{argument} \\
        \addlinespace[0.5em]

        Existential Generalization &
        \begin{argument}
            \premise{P(c) \text{ for some element of } c}
            \conclusion{\exists x P(x)}
        \end{argument} \\
        \addlinespace[0.5em]
        \bottomrule
    \end{tabular}
    \label{tab:quantifier-instantiation}
    \caption{Quantifier Instantiations}
\end{table}

% TODO: Truth Set
\begin{definition}[Truth Set]\label{def:truth-set}
    For a predicate $P(x)$, and domain $D$, the set of all elements of $D$ 
    for which $P(x)$ holds true is known as the truth set.
    \begin{equation*}
        \{ x \in D \mid P(x) \}
    \end{equation*}
\end{definition}

\begin{definition}[Universal Quantified Statements]\label{def:universal-quantified-stmts}
    For a predicate $Q(x)$, domain $D$, a universal statement is of the form:
    \begin{equation*}
        \forall x \in D \mid Q(x)
    \end{equation*}

    This statement is true if and only if the predicate $Q(x)$ is true for every element of $D$.
    If even one counterexample is found, then the statement is false.
\end{definition}

\begin{definition}[Existential Quantified Statements]\label{def:existential-quantified-stmts}
    For a predicate $Q(x)$, domain $D$, an existential statement is of the form:
    \begin{equation*}
        \exists x \in D \mid Q(x)
    \end{equation*}

    This statement is true if and only if there exists a single element for which the 
    predicate $Q(x)$ is true. It is false if no elements exist that uphold the predicate.
\end{definition}

Variable binding is the restriction of a variable to its quantifier.

Variable scope refers to the lifetime of a given variable.

\begin{definition}[Negation of Universal Statements]\label{def:neg-universal-stmts}
    \begin{equation*}
        \lsim (\forall x \in D, Q(x)) \equiv \exists x \in D \text{ such that } \lsim Q(x)
    \end{equation*}
\end{definition}

\begin{definition}[Negation of Existential Statements]\label{def:neg-existential-stmts}
    \begin{equation*}
        \lsim (\exists x \in D, Q(x)) \equiv \forall x \in D \text{ such that } \lsim Q(x)
    \end{equation*}
\end{definition}

\begin{definition}[Universal Instantiation]\label{def:universal-instantiation}
    If a property is true for all elements of a set, then it is true of any element taken from the set.
\end{definition}

\begin{bbox}{\centering Universal Modus Ponens}
    \begin{center}
        \begin{argument}
        \premise{\forall x, \text{ if } P(x) \text{ then } $Q(x)$}
        \premise{$P(a)$ \text{ for a particular } $a$}
        \conclusion{$Q(a)$}
    \end{argument}
    \end{center}
\end{bbox}

\begin{bbox}{\centering Universal Modus Tollens}
    \begin{center}
        \begin{argument}
        \premise{\forall x, \text{ if } P(x) \text{ then } $Q(x)$}
        \premise{\lsim Q(a) \text{ for a particular } a}
        \conclusion{\lsim P(a)}
    \end{argument}
    \end{center}
\end{bbox}

\section{Elementary Number Theory}

\subsection{Number Sets}

\begin{table}[!htbp]
    \centering
		\rowcolors{2}{}{gray!10}
    \begin{tabular}{ c c l l }
        \toprule
        Symbol  & Alternative Symbol    & Name      & Domain \\
        \midrule
        \Z      &                       & Integers  & $\{ ..., -3,-2,-1,0,1,2,3, ... \}$ \\
        \N      & $\Z^+$                & Naturals  & $\{ 1,2,3, ... \}$ \\
        \W      & $\Z^{\text{nonneg}}$  & Whole     & $\{ 0,1,2,3, ... \}$ \\
        \Q      &                       & Rationals & $\{ ..., -\frac{1}{2}, -1, 0, \frac{1}{3}, 1, ... \}$ \\
        \R      &                       & Reals     & $\{ ..., -\frac{1}{2}, -1, 0, \frac{1}{3}, 1, \sqrt{5}, ... \}$ \\
        \bottomrule
    \end{tabular}
    \label{tab:tbl-number-sets}
    \caption{Number Sets}
\end{table}

\subsection{Integers}

\begin{definition}[Even]\label{def:even}
    An integer $n$ is even if, and only if $n$ equals twice some integer.
    \begin{equation*}
        n \text{ is even } \iff n = 2k \text{ for some integer } k
    \end{equation*}
\end{definition}

\begin{definition}[Odd]\label{def:odd}
    An integer $n$ is odd if, and only if $n$ equals twice some integer plus 1.
    \begin{equation*}
        n \text{ is odd } \iff n = 2k + 1 \text{ for some integer } k
    \end{equation*}
\end{definition}

\begin{definition}[Prime]\label{def:prime}
    An integer $n$ is prime if, and only if $n > 1$ for all positive integers $r$, and $s$.
    If $n = rs$ then $r$ or $s$ equals $n$.
    \begin{equation*}
    \forall r,s \in Z^+, (n = rs) \to ((r = 1) \land (s = n)) \lor ((r = n) \land (s = 1))
    \end{equation*}
\end{definition}

\begin{definition}[Composite]\label{def:composite}
    An integer $n$ is composite if, and only if $n > 1$ for all positive integers $r$, and $s$.
    If $n = rs$ then $r$ or $s$ equals $n$ with $1 < r < n$ and $1 < s < n$.
    \begin{equation*}
        \exists r,s \in Z^+ \text{ such that } (n = rs) \land (1 < r < n) \land (1 < s < n)
    \end{equation*}
\end{definition}

\subsection{Number Systems}

\subsubsection{Binary}

Use the following table to write decimal numbers as the sum of powers of 2.

\begin{table}[!htbp]
    \centering
    \rowcolors{2}{}{cyan!10}
    \begin{tabular}{ *{12}{c} }
        %\toprule
        \arrayrulecolor{cyan}\toprule
        \textcolor{cyan}{\textbf{Power of 2}}      & $2^{10}$ & $2^9$ & $2^8$ & $2^7$  & $2^6$ & $2^5$ & $2^4$ & $2^3$ & $2^2$ & $2^1$ & $2^0$ \\
        \midrule
        \textcolor{cyan}{\textbf{Decimal Form}}    & 1024     & 512   & 256   & 128    & 64    & 32    & 16    &   8   & 4     & 2     & 1 \\
        \bottomrule
    \end{tabular}
    \label{tab:tbl-base-2-powers}
    \caption{Table of various powers of 2 for Binary and Decimal Conversions}
\end{table}

\subsubsection{Two's Complement}

\subsubsection{Signed Binary Integers}

\subsubsection{Binary Addition \& Subtraction}

Carrying still works in both binary addition and subtraction.

\begin{equation*}
    \begin{array}{B3}
        1_2 & \\
        {} + 1_2 & \\\hline
        10_2 \\
        
    \end{array}
\end{equation*}

\subsubsection{Hexadecimal}

% TODO:
% Fill in the rest of the missing sections:
\section{Proofs}

\subsection{Proof by Construction}
\begin{definition}[Proof by Construction]\label{def:proof-by-construction}
    A statement of the form "\qstmt{$\exists x$ }{D}{$Q(x)$}" is true
    if, and only if,
\end{definition}

\subsection{Proof by Counterexample}
\begin{definition}[Proof by Counterexample]\label{def:proof-by-counterexample}
    To disprove a statement of the form "$\forall x \in D$ if $P(x)$ then $Q(x)$",
    find a value of $x$ (the counterexample) in $D$ for which the hypothesis $P(x)$
    is true, and the conclusion $Q(x)$ is false.
\end{definition}

% Reductio Ad Absurdum
\subsection{Proof by Exhaustion}
\begin{definition}[Proof by Exhaustion]\label{def:proof-by-exhaustion}
    To prove a statement by exhaustion, simply list out all the possible
    cases for the given statement. Then evaluate each case for its truth value. 
\end{definition}

\subsection{Direct Proof}
\begin{definition}[Direct Proof]\label{def:direct-proof}\

    \begin{enumerate}
        \item Express the statement to be proved in the form "\qistmt{For every}{D}{$P(x)$}{$Q(x)$}"
        \item Suppose $x$ is a particular but arbitrarily chosen element of $D$,
            for which the hypothesis $P(x)$ is true.
        \item Show that the conclusion $Q(x)$ is true by using definitions,
            previously established results, and the rules for logical inference.
    \end{enumerate}

    Note:
    \begin{itemize}
        \item The first step is often done mentally.
        \item The suppose step is often abbreviated: "Suppose $x \in D$ and $P(x)$"
    \end{itemize}
\end{definition}

\subsection{Mathematical Induction}

\begin{definition}[Induction]\label{def:induction}
    Let $P(n)$ be a property that is defined for integers $n$, and $a$ is 
    a fixed integer constant. Suppose the following two statements are true:
    \begin{enumerate}
        \item $P(a)$ is true.
        \item For every integer $k \geq a$ if $P(a)$ is true then $P(k+1)$ is true.
    \end{enumerate}
    
    Then the statement 
    \begin{equation*}
        \text{ for every integer $n \geq a$, $P(n)$ }
    \end{equation*}
\end{definition}

To prove by mathematical induction, you need to follow these steps.

\begin{definition}[Method of Proof]\label{def:induction-steps}
    \begin{enumerate}
        \item Show that $P(a)$ is true.
        \item Show that for every integer $k \geq a$ if $P(k)$ is true then
            $P(k+1)$ is true. To perform this, suppose the inductive hypothesis
            $P(k)$ is true for any arbitrarily chosen integer $k \geq a$. 
            Then show that $P(k+1)$ is true.
    \end{enumerate}
\end{definition}

\begin{definition}[Sum of the First $n$ Integers]\label{def:sum-of-n-integers}
    For every integer $n \geq 1$
    \begin{equation*}
        1 + 2 + ... + n = \frac{n(n+1)}{2}
    \end{equation*}
\end{definition}

\begin{definition}[Closed Form]\label{def:closed-form}
    If a sum with a variable number of terms is shown to equal an expression
    that does not contain either an ellipsis or a summation symbol, then the
    expression is written in closed form.
\end{definition}

% TODO:
% Fill in the rest of the missing sections:
\section{Relations}

% Relations
% Arrow Diagrams

%   Exponential Functions
%   Logarithmic Functions
%   Function Composition

\begin{definition}[Relation]\label{def:relation}
    A relation from $A$ to $B$, of two sets $A$, $B$, is a subset of $A \times B$.
    Given an ordered pair $(x, y)$ in $A \times B$, $x$ is related to $y$ by $\rel$,
    denoted $x \rel y$, if and only if $(x, y)$ is in $\rel$. 
    
    In other words, for a relation:

    \begin{equation*}
        x \rel y \text{ means that } (x,y) \in \rel
    \end{equation*}

    $x$ is not related to $y$ by $\mathrel{R}$ if 

    \begin{equation*}
        x \nrel y \text { means that } (x,y) \notin \rel
    \end{equation*}

    Note that for a relation, the set $A$ is the domain of
    $\rel$ and the set $B$ is its co-domain.
    
\end{definition}

\subsection{Arrow Diagrams}
% TODO:

\subsection{Inverse Relations}

\begin{definition}[Inverse Relation]\label{def:inverse-relation}
    Let $R$ be a relation from $A$ to $B$. Define the inverse relation $R\inv$ from $B$ to $A$ as:
    \begin{equation*}
        R\inv = \{(y, x) \in B \times A \mid (x, y) \in \R \}
    \end{equation*}
    
    This can be written operationally as:
    \begin{equation*}
        \text{For all $x \in A$ and $y \in B$, } (y, x) \in R\inv \iff (x,y) \in R
    \end{equation*}
\end{definition}

\subsection{N-ary Relations}
\begin{definition}[N-ary Relations]\label{def:n-ary-relations}
    Given sets $A_1, A_2, ..., A_n$, an n-ary relation $R$ on $A_1 \times A_2 \times ... A_n$
    is a subset of $A_1 \times A_2 \times ... A_n$. The special cases of 2-ary, 3-ary, 4-ary
    relations are called binary, ternary, quaternary relations.
\end{definition}

\subsection{Properties of Relations}

\begin{definition}[Reflexivity]\label{def:reflexivity}
    Let $R$ be a relation on a set $A$.
    \bl
    $R$ is reflexive if, and only if, for every $x \in A$, $x \rel x$.

    Equivalently:
    \begin{equation*}
        \text{$\rel$ is reflexive} \iff \text{ for every $x \in A$, $(x,x) \in \rel$}
    \end{equation*}

    Informally: A reflexive relation is one where:
    \begin{align*}
        \text{Each element is related to itself}
    \end{align*}
 \end{definition}

\begin{definition}[Symmetry]\label{def:symmetry}
    Let $R$ be a relation on a set $A$.
    \bl 
    $R$ is symmetric if, and only if, for every $x,y \in A$, if $x \rel y$ then $y \rel x$.

    Equivalently:
    \begin{equation*}
        \text{$\rel$ is symmetric} \iff \text{ for every $x,y \in A$, if $(x,y) \in \rel$ then $(y,x) \in \rel$}
    \end{equation*}

    Informally, a symmetric relation is one where:
    \begin{align*}
        \text{If } &\text{any one element is related to any other element,} \\
                   &\text{then the second element is related to the first.} \\
    \end{align*}
\end{definition}

\begin{definition}[Transitive]\label{def:transitive}
    Let $R$ be a relation on a set $A$.
    \bl
    $R$ is transitive if, and only if, for every $x,y,z \in A$, if $x \rel y$ and $y \rel z$ then $x \rel z$.

    Equivalently:
    \begin{equation*}
        \text{$\rel$ is transitive} \iff \text{ for every $x,y,z \in A$, if $(x,y) \in \rel$ and $(y,z) \in \rel$ then $(x,z) \in \rel$}
    \end{equation*}

    Informally, a transitive relation is one where:
    \begin{align*}
        \text{If } & \text{any one element is related to a second} \\
                   & \text{and the second element is related to a third} \\
                   & \text{then the first element is related to the third} \\
    \end{align*}
\end{definition}

% TODO:
% Fill in the rest of the missing sections:
\section{Functions}

\begin{definition}[Function]\label{def:function}
    A function $F$ from sets $A$ to $B$ is relation with domain $A$ and co-domain $B$
    that satisfies the following two properties.
    
    \begin{enumerate}
        \item For every element $x$ in $A$ there is an element $y$ in $B$ such that $(x,y) \in F$
        \item For all elements $x$ in $A$ and $y$ and $z$ in $B$,
            \begin{equation*}
                \text{if } (x,y) \in F \text{ and } (x,z) \in F, \text{ then } y = z
            \end{equation*}
    \end{enumerate}

    In other words, a relation $F$ from $A$ to $B$ is a function if and only if
    \begin{enumerate}
        \item Every element of $A$ is an ordered pair of $F$
        \item No two distinct pairs in $F$ have the same element.
    \end{enumerate}
    
    For any element $x$ of $A$, if $F$ is a function then $F(x)$ denotes
    the unique element in $B$ related by $x$, read as "$F$ of $x$".
\end{definition}

\subsection{Function Machines}

\begin{definition}[Squaring Function]\label{def:sqr-fn}
    \begin{equation*}
        \fn{f}{x}{x^2}
    \end{equation*}
    from $\R$ to $\R$
\end{definition}

\begin{definition}[Successor Function]\label{def:succ-fn}
    \begin{equation*}
        \fn{g}{n}{n + 1}
    \end{equation*}
    from $\Z$ to $\Z$
\end{definition}

\begin{definition}[Constant Function]\label{def:const-fn}
    \begin{equation*}
        \fn{h}{r}{n}
    \end{equation*}
    from $\Q$ to $\Z$, where $n$ is any constant number.
\end{definition}

\begin{definition}[Function Equality]\label{def:function-equality}
    Given two functions $f$, $g$ from sets $A$ to $B$ :
    \begin{equation*}
        f = \{ (x,y) \in A \times B \mid y = f(x) \} \text{ and } g = \{ (x,y) \in A \times B \mid y = g(x) \
    \end{equation*}
    
    Then:
    \begin{equation*}
        f = g \text{ if and only if } f(x) = g(x) \text{ for all $x$ in $A$ }
    \end{equation*}
\end{definition}

\subsection{One-To-One Functions}
\begin{definition}[One-To-One]\label{def:one-to-one}
    Let $F$ be a function from set $X$ to set $Y$. $F$ is one-to-one (injective) if
    and only if, for all elements $x_1$, $x_2$ in $X$:
    \begin{enumerate}
        \item if $F(x_1) = F(x_2)$, then $x_1 = x_2$
        \item if $x_1 \neq x_2$, then $F(x_1) \neq F(x_2)$
    \end{enumerate}

    Formally:
    \begin{equation*}
        \fn{F}{X}{Y} \text{ is one-to-one} \iff \forall x_1,x_2 \in F(x_1) = F(x_2), \text{ then } x_1 = x_2
    \end{equation*}
\end{definition}

\begin{definition}[Determining One-To-One Functions]\label{def:det-one-to-one}
    A function is not one-to-one if:
    \begin{equation*}
        \fn{F}{X}{Y} \text{ is not one-to-one} \iff \exists x_1,x_2 \in X \text{ such that } F(x_1) = F(x_2) \text{ and } x_1 \neq x_2
    \end{equation*}

    In other words, if elements $x_1$, and $x_2$ can be found that have
    the same function value but are not equal, then $F$ is not one-to-one.
\end{definition}

To prove a function is one-to-one on an infinite set, we will generally use the method of direct proof.

\begin{definition}[Proving One-To-One Functions on Infinite Sets]\label{def:prove-one-to-one}
    \textbf{Suppose:} $x_1$, $x_2$ are both elements of $X$ such that $F(x_1) = F(x_2)$

    \textbf{Show:} $x_1 = x_2$

    To show that the function is not one-to-one we will instead show:

    \text{Show:} Find elements $x_1$ and $x_2$ in $X$ such that $F(x_1) = F(x_2)$ but $x_1 \neq x_2$
\end{definition}

\subsection{Onto Functions}
\begin{definition}[Onto]\label{def:onto}
    Let $F$ be a function from $X$ to $Y$. F is onto (surjective) if, and only if,
    given any element $y$ in $Y$, it is possible to find an element $x$ in $X$ with
    the property that $y = F(x)$
    \begin{equation*}
        F\colon X \to Y \text{ is onto } \iff \forall y \in Y, \exists x \in X \text{ such that } F(x) = y
    \end{equation*}

    In other words
    \begin{equation*}
        F\colon X \to Y \text{ is not onto } \iff \exists y \in Y \text{ such that } \forall x \in X F(x) \neq y
    \end{equation*}
\end{definition}

To prove a function is one-to-one on an infinite set, we will generally use the method of generalizing from the generic particular.

\begin{definition}[Proving Onto Functions on Infinite Sets]\label{def:prove-onto}
    \textbf{Suppose:} $y$ is any element of $Y$

    \textbf{Show:} $\exists x \in X$ such that $F(x) = y$

    To show that the function is not onto we will instead show:

    \text{Show:} Find an element $y \in Y$ such that $y \neq F(x)$ for any $x \in X$
\end{definition}

\subsection{Exponential Functions}
\subsection{Logarithmic Functions}

\subsection{One-To-One Correspondences}
\begin{definition}[One-To-One Correspondence]\label{def:one-to-one-correspondence}
    A one-to-one correspondence (bijection) from sets $X$ to $Y$ is a function $\fn{F}{X}{Y}$
    that is both one-to-one and onto.
\end{definition}

\subsection{Inverse Functions}

\begin{definition}[Inverse Function]\label{def:inverse-fn}
    Suppose $\fn{F}{X}{Y}$ is a one-to-one correspondence. Then there is a function $\fn{F\inv}{Y}{X}$ that is defined as:

    Given any element $y \in Y$,
    \begin{equation*}
        F\inv(y) = \text{ the unique element $x \in X$ such that $F(x) = y$ }
    \end{equation*}
    or more formally:
    \begin{equation*}
        F\inv(y) = x \iff y = F(x)
    \end{equation*}
    
    $F\inv$ is known as the inverse function for $F$.
\end{definition}

\begin{theorem}[Inverse Function Properties]\label{def:inverse-fn-props}
    If $X$ and $Y$ are sets and $\fn{F}{X}{Y}$ is one-to-one and onto,
    then $\fn{F\inv}{Y}{X}$ is also one-to-one and onto
\end{theorem}

\section{Sequences}

\begin{definition}[Sequence]\label{def:sequence}
    A \textbf{sequence} is a function whose domain is either 
    \begin{enumerate}
        \item All the integers between two given integers.
        \item All the integers greater than or equal to a given integer.
    \end{enumerate}
    

    
\end{definition}

\subsection{Sum}

\begin{definition}[Sum]\label{def:sum}
    If $m$ and $n$ are integers and $m \leq n$, the summation
    from $k$ equals $m$ of $a$-sub-$k$, denoted$\sum_{k=m}^n a_k$,
    is the sum of all the terms $a_m, a_{m+1} a_{m+2}, ..., a_n$.
    \begin{equation*}
        \sum_{k=m}^n a_k = a_m, a_{m+1} a_{m+2}, ..., a_n
    \end{equation*}
    Where
    \begin{itemize}
        \item $k$ is the index of the sum. Also known as the sum's subscript.
        \item $m$ is the lower limit of the sum
        \item $n$ is the upper limit of the sum
        \item $a_m, a_{m+1} a_{m+2}$ and $a_n$ are the sum's terms.
        \item $a_m, a_{m+1} a_{m+2}, ..., a_n$ is the expanded form of the sum.
    \end{itemize}

    Note that an open ended sequence such as $a_m, a_{m+1}, a_{m+2}, ...$,
    the sequence is known as an infinite sequence.
\end{definition}

\subsection{Product}

\begin{definition}[Product]\label{def:product}
    If $m$ and $n$ are integers and $m \leq n$,
    the product from $k$ equals $m$ to $n$ of $a$-sub-$k$ denoted
    $\prod_{k=m}^n a_k$
    is the product of the terms $a_m \cdot a_{m+1} \cdot a_{m+2} \cdot ... \cdot a_n$
    \begin{equation*}
        \prod_{k=m}^n a_k = a_m \cdot a_{m+1} \cdot a_{m+2} \cdot ... \cdot a_n
    \end{equation*}

    The Greek letter used is (Pi $\Pi$).
\end{definition}

\begin{definition}[Product (Recursive Definition)]\label{def:recursive-product}
    If $m$ is any integer
    \begin{equation*}
        \prod_{k=m}^m = a_m \text{ and } \prod_{k=m}^n a_k = \Biggl( \prod_{k=m}^{n-1} \Biggl) \cdot a_n \text { for every integer $n > m$ }
    \end{equation*}
\end{definition}

\subsection{Sum \& Product Properties}

\begin{definition}[Sum \& Product Properties]\label{def:props-sums-prods}
    If $a_m, a_{m+1}, a_{m+2}, ...$ and $b_m, b_{m+1}, b_{m+2}, ...$
    are sequences of real numbers and $c$ is any real number then
    the following operations hold for any integer $n \geq m$.

    \begin{align*}
        \text{1.} & \sum_{k=m}^n a_k + \sum_{k=m}^n b_k = \sum_{k=m}^n (a_k + b_k) \\
        \text{2.} & \text{ } c \cdot \sum_{k=m}^n a_k = \sum_{k=m}^n c \cdot a_k \\
        \text{3.} & \Biggl( \prod_{k=m}^n a_k \Biggl) \cdot \Biggl( \prod_{k=m}^n b_k \Biggl) =
                    \prod_{k=m}^n (a_k \cdot b_k) \\
    \end{align*}
\end{definition}

\subsection{Change of Variable}

The variable used for the index in sums or products can be replaced so long
as every occurrence of the variable is also replaced.

\subsection{Common Sequences}

\subsubsection{Factorial}
\begin{definition}[Factorial]\label{def:factorial}
    For every positive integer $n$, $n$ factorial denoted $n!$,
    is the product of all the integers from 1 to $n$.
    \begin{equation*}
        n! = n \cdot (n - 1) \cdot ... \cdot 3 \cdot 2 \cdot 1
    \end{equation*}
    
    Zero factorial is defined to be 1:
    \begin{equation*}
        0! = 1
    \end{equation*}
\end{definition}

\begin{definition}[Factorial (Recursive)]\label{def:factorial-recursive}
    \begin{equation*}
        n! = 
        \begin{cases}
            1  & \text{ if $n = 0$ } \\
            n \cdot (n - 1)! & \text{ if $n \geq 1$}
        \end{cases}
    \end{equation*}
\end{definition}

\subsection{Combinatorics}

\begin{definition}[Choose Notation]\label{def:choose}
    For integers $n$ and $r$ such that $0 \leq r \leq n$.

    \begin{equation*}
        {n \choose r}
    \end{equation*}

    Read "$n$ choose $r$" represents the number of subsets of size $r$ that
    can be chosen from a set of $n$ elements.
\end{definition}

\begin{definition}[Combinations Formula]\label{def:combinations}
    For all integers $n$ and $r$ with $0 \leq r \leq n$

    \begin{equation*}
        {n \choose r} = \dfrac{n!}{r!(n - r)!}
    \end{equation*}
\end{definition}

\end{document}
