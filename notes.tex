\documentclass[11pt]{article}

% Imports
\usepackage[table]{xcolor}
\usepackage{coco}
\usepackage[numbersets,logic]{shortmath}
\usepackage{thmtools}
\usepackage{wrapfig}
%\usepackage{parskip}

% Make better use of the page
% TODO: We'll keep margins like this until we add in margin notes
\geometry{
    left=2cm,
    top=2cm,
    right=2cm,
    bottom=2cm,
}

% ==== Markdown-like bold styling ====
% Usage:
% \StarON
% **This will be bolded** 
% \StarOFF % Disable markdown styling
\makeatletter
\def\starparse{\@ifnextchar*{\bfstarx}{\itstar}}
\def\bfstarx#1{\@ifnextchar*{\bfitstar\@gobble}{\bfstar}}
\makeatother
\def\itstar#1*{\textit{#1}\starON}
\def\bfstar#1**{\textbf{#1}\starON}
\def\bfitstar#1***{\textbf{\textit{#1}}\starON}
\def\starON{\catcode`\*=\active}
\def\starOFF{\catcode`\*=12}
\starON
\def*{\starOFF \starparse}
\starOFF

% ==== Theorem Styles ====
% Theorem environments are  : definition,
% Theorem styles are        :
    
% Declare theorem styles
%\theoremstyle{definition}

%\declaretheoremstyle[headfont=\bfseries\sffamily, bodyfont=\normalfont, mdframed={nobreak}]{thmgreenbox}
%\declaretheorem[numberwithin=section, style=thmgreenbox , name=Definition]{definition}

%\theoremstyle{definition}
%\newtheorem{definition}{Definition}[section]

% Custom Name Definitions
\declaretheoremstyle[
    spaceabove=\topsep,
    spacebelow=\topsep,
    headfont=\bfseries\sffamily,
    bodyfont=\normalfont,
    notefont=\bfseries, notebraces={}{},
    postheadspace=0.5em,
    name={\ignorespaces},
    numbered=no,
    mdframed={nobreak}, % Make boxed
    headpunct=:
]{definition}
\declaretheorem[style=definition]{definition}

% TODO: Create theorem style where the name of the thing is actually shown

% TODO: Remove disgusting paragraph indent
\setlength{\lineskip}{0.0pt}
\setlength{\lineskiplimit}{0pt}
\setlength{\parskip}{0pt}
\setlength{\parindent}{0pt}
\setlength{\baselineskip}{0pt}

\begin{document}

% TODO: Make nice page
% TODO: Refactor into a cover page

% Cover Page
\begin{titlepage}
    \begin{center}
        \vspace*{1cm}
        \huge \textbf{Discrete Mathematics}

        \vspace*{1.5em}

        \Large{Joseph M. R. Diza}

        \vfill
    \end{center}
\end{titlepage}
\newpage

% TODO: Add fancyhdr shit here

% Table of Contents
\tableofcontents

% Quick Introduction
\section{Introduction}

This document contains various definitions, laws, notation for Discrete Mathematics.

The content is stripped of many examples for brevity purposes, and is essentially a reference book.

% Content
\section{Logical Statements}

\begin{definition}[Propositions]\label{def:props}
Declarative true or false (but not both) statements.
\end{definition}

\begin{definition}[Primitive Statements]\label{def:prims}
Propositions that are in their simplest terms.
\end{definition}

\begin{definition}[Universal Statements]\label{def:universal-stmts}
Statement that a property holds true for all elements in a set.
\end{definition}

\begin{definition}[Conditional Statements]\label{def:cond-stmts}
Statement that if a condition is true then another condition is true.
\end{definition}

\begin{definition}[Existential Statements]\label{def:exist-stmts}
Statement that is true for at least one element in a set, given a property is a statement.
\end{definition}

\section{Set Theory}

Set-Builder notation (also known as Set Roster Notation) is used to describe sets.

\begin{definition}[Set]\label{def:set}
An arbitrary collection of elements.
\end{definition}

\subsection{Set Roster Notation}
\begin{table}[!htbp]
    % Symbol        ,     Name      , Meaning           | Usage
    % {, }          , Curly Braces  , The set of all    | Denotes a set
    % |             , Pipe/Bar      , Such that         | Indicates conditions on sets
    % ...           , Ellipses      , And so forth      | Elides numbers from the entire set for brevity
    \centering
    \caption{Set Roster Notation}
		\rowcolors{2}{}{gray!10}
    \begin{tabular}{ c l l l }
        \toprule
        Symbol  & Name          & Meaning           & Usage \\
        \midrule
        \{ \}   & Curly Braces  & The set of all    & Denotes a set \\
        |       & Pipe/Bar      & Such that         & Indicates conditions on sets \\
        ...     & Ellipses      & And so forth      & Elides numbers from the entire set for brevity \\
        \bottomrule
    \end{tabular}
    \label{tab:tbl-set-notation}
\end{table}

\subsection{Common Number Sets}
% TODO: Add sets for the groups of numbers
% TODO: Develop set notation functions to make writing easier
% TODO: Fix table headers not being centered

% TODO: Wrap the text on this
\begin{table}[!htbp]
    % Symbol        ,   Name        , Domain
    % Z             , Integers      , 
    % Q             , Rationals     , 
    % R             , Reals         , 
    \centering
		\rowcolors{2}{}{gray!10}
    \begin{tabular}{ c l l }
        \toprule
        Symbol  & Name      & Set \\
        \midrule
        \Z      & Integers  & $\{-3,-2,-1,0,1,2,3, ... \}$ \\
        \Q      & Rationals & $\{ -1/2, 0.333, ... \}$ \\
        \R      & Reals     & $\{ \sqrt{5} \}$ \\
        \bottomrule
    \end{tabular}
    \label{tab:tbl-number-sets}
    \caption{Common Number Sets}
\end{table}

\starON
**Note:** The $^+$ superscript denotes positive number elements in these sets. 

**Note:** The $^{nonneg}$ superscript denotes non-negative numbers (0, 1, 2, 3, ...). 
\starOFF

\subsection{Set Theory Notation}
\begin{table}[!htbp]
    \centering
        \rowcolors{2}{}{gray!10}
        \begin{tabular}{ c l l l }
        \toprule
        Symbol      & LaTeX             & Name              & Meaning \\
        \midrule
        $\in$       & \verb|\in|        & In                & Is an element of ... \\
        $\notin$    & \verb|\notin|     & Not in            & Is not an element of ... \\
        $\subset$   & \verb|\subset|    & Proper Subset     & Is a proper subset of ... \\
        $\subseteq$ & \verb|\subseteq|  & Subset            & Is a subset of ... \\
        $\subset$   & \verb|\subset|    & Proper Superset   & Is a proper superset of ... \\
        $\subseteq$ & \verb|\subseteq|  & Superset          & Is a superset of ... \\
        $\cup$      & \verb|\cup|       & Set Union         & The set of elements in either set A or B \\
        $\cap$      & \verb|\cap|       & Set Intersection  & The set of elements in both set A and B \\
        $\setminus$ & \verb|\setminus|  & Set Difference    & The set of elements of A that are not in B \\
        \bottomrule
    \end{tabular}
    \label{tab:tbl}
    \caption{}
\end{table}


\newpage

\begin{definition}[Subset]\label{def:subset-txt}
If A, B are sets, then A is a subset of B if and only if every element of A is an element of B
\end{definition}

% TODO Find symbols for this shit
%\begin{definition}[Subset]\label{def:subset}
    %Let A, B be sets,
        %A subset B iff for all elem a in A, a is in B
%\end{definition}

% TODO: Add symbolic variant definitions for sets, distinguish with text variants with -txt

Note that 'contained-in' are alternative ways of saying 'is a subset of'.

\begin{definition}[Proper Subset]\label{def:proper-subset-txt}
Let A, B be sets, A is a proper subset of B if and only if every element of A is in B, but there
is at least one element of B not in A.
\end{definition}

% TODO: Cartesian product def
% TODO: Ordered pair def
% TODO: N-tuples def
% TODO: Strings def

\section{Predicate Logic}

\subsection{Logical Operators}
\begin{table}[!htbp]
    \centering
    \caption{Logical Operators}
		\rowcolors{2}{}{gray!10}
    \begin{tabular}{ c c l l l l  }
        \toprule
        Symbol      & Programming   & LaTeX         & Name          & Meaning \\
        \midrule
        $\land$     & \&\&          & \verb|\wedge| & And           & Conjunction of two terms \\
        $\lor$      & ||            & \verb|\lor|   & Or            & Disjunction of two terms \\
        $\lxor$     & \string^      & \verb|\lxor|  & Xor           & Exclusive logical or \\
        $\neg$      & !             & \verb|\neg|   & Not           & Negation of a term \\
        $\limp$     &               & \verb|\to|    & Implies       & Logical implication \\
        $\iff$      &               & \verb|\iff|   & Biconditional & Logical Equivalence \\
        %\land           & \&\&                  & \wedge
        \bottomrule
    \end{tabular}
    \label{tab:tbl-logic-ops}
\end{table}

Note: An alternative syntax for negation is tilde: $\sim$ (\verb|\sim|).

Note: In the biconditional, the hypothesis implies the conclusion and vice versa.

% TODO: Logical operators
% TODO: Table showing LaTeX symbols
% TODO: Add notes section to briefly describe these things

% Logical Symbol    , Programming Symbol    , LaTeX Symbol  , Name          , Meaning
% &&                , &&                    , \wedge        , Logical And   , Conjunction of two statements
% ||                , ||                    , \lor          , Logical Or    , Disjunction of two statements
% Not               , !                     , \lnot         , Negation      , Negates a statement
% Xor               , xor                   , \xor          , Exclusive Or  , Logical Or, but both statements cannot be true

% TODO: Truth values, Truth tables.

% TODO: Definitions
% TODO: negation
% TODO: conjunction
% TODO: disjunction

% TODO: Short bit about exclusive or
% TODO: Logical equivalence def
% TODO: Add short algorithmic bit about testing for logical equivalence

% TODO: Laws of Logic
% TODO: Tautology, Contradiction definitions

\subsection{Conditional Statements}
% TODO: Hypothesis, conclusion, implication
% TODO: Truth table for p -> q
% TODO: Definition of p -> q

% TODO: Negation Contrapositive, Converse, Inverse defs
% TODO: Biconditional def

% TODO: Necessary & sufficient conditions.

\subsection{Arguments}

% TODO: Arguments definition: Page: 66

% TODO: Rules of Inference

TODO: Entire section of gotchas

\subsection{Quantifiers}

\begin{table}[!htbp]
    \centering
    \begin{tabular}{ *{2}{cc} }
        \toprule
        Symbol      & LaTeX             & Name                      & Meaning \\
        \midrule
        $\exists$   & \verb|\exists|    & Existential Quantifier    & There exists at least one \\
        $\forall$   & \verb|\forall|    & Existential Quantifier    & For all \\
        \bottomrule
    \end{tabular}
    \label{tab:tbl-quantifiers}
    \caption{3}
\end{table}

% TODO: Universal, Existential quantifiers definitions, and table meanings

% Table of quantifiers
% Symbol    , Name          , Meaning
% ∀         , Universal     , For all elements of the set
% E         , Existential   , There exists an element of the set

Variable binding is the restriction of a variable to its quantifer.

Variable scope refers to the lifetime of a given variable.

% TODO: Brief description, formal definition of implicit quantification

% Negation of a universal, existential statement
% Order of quantifiers

% TODO: Instantiation
% Universal instantiation
    
\end{document}
