\documentclass{article}

% Imports
\usepackage{coco}
\usepackage{thmtools}

\geometry{
    left=2cm,
    top=2cm,
    right=2cm,
    bottom=2cm,
}

% Notes:
% - The 'txt' suffix is used to describe a definition defined with natural sentences.

% ==== Theorem Styles ====
% Theorem environments are  : definition,
% Theorem styles are        :
    
% Declare theorem styles
%\theoremstyle{definition}

%\declaretheoremstyle[headfont=\bfseries\sffamily, bodyfont=\normalfont, mdframed={nobreak}]{thmgreenbox}
%\declaretheorem[numberwithin=section, style=thmgreenbox , name=Definition]{definition}

%\theoremstyle{definition}
%\newtheorem{definition}{Definition}[section]

\declaretheoremstyle[
    spaceabove=\topsep,
    spacebelow=\topsep,
    headfont=\normalfont\bfseries,
    notefont=\bfseries, notebraces={}{},
    %bodyfont=\normalfont\itshape,
    bodyfont=\normalfont,
    postheadspace=0.5em,
    name={\ignorespaces},
    numbered=no,
    headpunct=:
]{definition}
\declaretheorem[style=definition]{definition}
%\newtheorem{definition}{Definition}[section]

%\declaretheoremstyle[
    %spaceabove=\topsep, spacebelow=\topsep,
    %headfont=\normalfont\bfseries,
    %notefont=\bfseries, notebraces={}{},
    %bodyfont=\normalfont\itshape,
    %postheadspace=0.5em,
    %name={\ignorespaces},
    %numbered=no,
    %headpunct=:
%]{definition}
%\declaretheorem[style=definition]{definition}

% TODO: Create theorem style where the name of the thing is actually shown

\begin{document}

% TODO: Make nice page

\tableofcontents

\section{Statements}

\begin{definition}[Propositions]\label{def:props}
Propositions are declarative true or false (but not both) statements.
\end{definition}

\begin{definition}\label{def:prims}
Primitive statements are propositions that are in their simplest terms.
\end{definition}

\begin{definition}\label{def:universal-stmts}
A universal statement states a property is true for all elements in a set.
\end{definition}

\begin{definition}\label{def:cond-stmts}
A conditional statement states that if a condition is true then another condition is true.
\end{definition}

\begin{definition}\label{def:exist-stmts}
An existential statement is true for at least one element in a set, given a property is a statement.
\end{definition}

\section{Sets}

Set-Builder notation is used to describe sets.

% TODO: Set definition
\begin{definition}\label{def:set}
A set is a collection of elements
\end{definition}

\begin{table}[!htbp]
    % Symbol        ,     Name      , Meaning           | Usage
    % {, }          , Curly Braces  , The set of all    | Denotes a set
    % |             , Pipe/Bar      , Such that         | Indicates conditions on sets
    % ...           , Ellipses      , And so forth      | Elides numbers from the entire set for brevity
    \centering
    \caption{Set Roster Notation}
    %\begin{tabular}{ *{5}{ll} }
    %\begin{tabular}{ c | l | l | l }
    \begin{tabular}{ c l l l }
        \toprule
        Symbol  & Name          & Meaning           & Usage \\
        \midrule
        \{ \}  & Curly Braces  & The set of all    & Denotes a set \\
        |       & Pipe/Bar      & Such that         & Indicates conditions on sets \\
        ...     & Ellipses      & And so forth      & Elides numbers from the entire set for brevity \\
        \bottomrule
    \end{tabular}
    \label{tab:tbl1}
\end{table}

\newpage

% TODO: Add sets for the groups of numbers
% TODO: Develop set notation functions to make writing easier
% TODO: Fix table headers not being centered
\begin{table}[!htbp]
    % Symbol        ,   Name        , Domain
    % Z             , Integers      , 
    % Q             , Rationals     , 
    % R             , Reals         , 
    \centering
    \begin{tabular}{ c l l }
        \toprule
        Symbol  & Name      & Set \\
        \midrule
        \Z      & Integers  & $\{-3,-2,-1,0,1,2,3, ... \}$ \\
        \Q      & Rationals & $\{ -1/2, 0.333, ... \}$ \\
        \R      & Reals     & $\{ \sqrt{5} \}$ \\
        \bottomrule
    \end{tabular}
    \label{tab:tbl2}
    \caption{Number Sets}
\end{table}


% TODO: Definitions for continuous and discrete


% TODO: Add quick table cheat sheet to use to describe notation
The '+' superscript is used in these sets to denote positive number elements.
The 'nonneg' superscript is used to denote non-negative numbers (positive numbers and zero).


\begin{definition}[subset-txt]\label{def:subset-txt}
If A, B are sets, then A is a subset of B if and only if every element of A is an element of B
\end{definition}

% TODO: Add symbolic variant definitions for sets, distinguish with text variants with -txt

Note that 'contained-in' are alternative ways of saying 'is a subset of'.

\begin{definition}[proper-subset-txt]\label{def:proper-subset-txt}
Let A, B be sets, A is a proper subset of B if and only if every element of A is in B, but there
is at least one element of B not in A.
\end{definition}


% TODO: Cartesian product def
% TODO: Ordered pair def
% TODO: N-tuples def
% TODO: Strings def

\section{Logic}

% TODO: Logical operators
% TODO: Table showing LaTeX symbols
% TODO: Add notes section to briefly describe these things

% Logical Symbol    , Programming Symbol    , LaTeX Symbol  , Name          , Meaning
% &&                , &&                    , \wedge        , Logical And   , Conjunction of two statements
% ||                , ||                    , \lor          , Logical Or    , Disjunction of two statements
% Not               , !                     , \lnot         , Negation      , Negates a statement
% Xor               , xor                   , \xor          , Exclusive Or  , Logical Or, but both statements cannot be true

% TODO: Truth values, Truth tables.

% TODO: Definitions
% TODO: negation
% TODO: conjunction
% TODO: disjunction

% TODO: Short bit about exclusive or
% TODO: Logical equivalence def
% TODO: Add short algorithmic bit about testing for logical equivalence

% TODO: Laws of Logic
% TODO: Tautology, Contradiction definitions

\subsection{Conditional Statements}
% TODO: Hypothesis, conclusion, implication
% TODO: Truth table for p -> q
% TODO: Definition of p -> q

% TODO: Negation Contrapositive, Converse, Inverse defs
% TODO: Biconditional def

% TODO: Necessary & sufficient conditions.

\subsection{Arguments}

% TODO: Arguments definition: Page: 66

% TODO: Rules of Inference

TODO: Entire section of gotchas

\subsection{Quantifiers}

% TODO: Universal, Existential quantifiers definitions, and table meanings

% Table of quantifiers
% Symbol    , Name          , Meaning
% ∀         , Universal     , For all elements of the set
% E         , Existential   , There exists an element of the set

Variable binding is the restriction of a variable to its quantifer.

Variable scope refers to the lifetime of a given variable.

% TODO: Brief description, formal definition of implicit quantification

% Negation of a universal, existential statement
% Order of quantifiers

% TODO: Instantiation
% Universal instantiation
    
\end{document}
