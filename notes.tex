\documentclass[11pt]{article}

% Imports
\usepackage[table]{xcolor}
\usepackage[noparindent]{coco}
\usepackage[numbersets,logic]{shortmath}
\usepackage{code}
\usepackage{discrete}
\usepackage{thmstyles}
\usepackage{boxstyles}

\usepackage{wrapfig}
\usepackage{setspace}
\usepackage{makecell}
\usepackage{adjustbox}
\usepackage{titlesec}

% Make better use of the page
% TODO: We'll keep margins like this until we add in margin notes
\geometry{
    margin=2cm,
}

\setlength{\lineskip}{0.0pt}
\setlength{\lineskiplimit}{0pt}
\setlength{\baselineskip}{0pt}

% Arial Font on Headings
\titleformat{\section}{\sffamily\upshape\bfseries\huge}{\thesection}{1em}{}
\titleformat{\subsection}{\sffamily\upshape\bfseries\large}{\thesubsection}{1em}{}
\titleformat{\subsubsection}{\sffamily\upshape\bfseries}{\thesubsubsection}{1em}{} 

\newcolumntype{L}[1]{>{\raggedright\let\newline\\\arraybackslash\hspace{0pt}}m{#1}}

\begin{document}

\begin{titlepage}
    \begin{center}
        \vspace*{1cm}
        \huge \textbf{Discrete Mathematics}

        \vspace*{1.5em}

        \Large{Joseph Diza}

        \vfill
    \end{center}
\end{titlepage}


% TODO: Add fancyhdr shit here

% Table of Contents
\tableofcontents

% Quick Introduction
\section{Introduction}

%\starON % Enable markdown style bolding

This document contains various definitions, laws, notation for Discrete Mathematics.

The content is stripped of many examples for brevity purposes, and is essentially a reference book.

% Content
\section{Logical Statements}

\begin{definition}[Propositions]\label{def:props}
Declarative true or false (but not both) statements.
\end{definition}

\begin{definition}[Primitive Statements]\label{def:prims}
Propositions that are in their simplest terms.
\end{definition}

\begin{definition}[Universal Statements]\label{def:universal-stmts}
Statement that a property holds true for all elements in a set.
\end{definition}

\begin{definition}[Conditional Statements]\label{def:cond-stmts}
Statement that if a condition is true then another condition is true.
\end{definition}

\begin{definition}[Existential Statements]\label{def:exist-stmts}
Statement that is true for at least one element in a set, given a property is a statement.
\end{definition}

\section{Set Theory}

Set-Builder notation (also known as Set Roster Notation) is used to describe sets.

\begin{definition}[Set]\label{def:set}
An arbitrary collection of elements.
\end{definition}

\subsection{Set Roster Notation}
\begin{table}[!htbp]
    % Symbol        ,     Name      , Meaning           | Usage
    % {, }          , Curly Braces  , The set of all    | Denotes a set
    % |             , Pipe/Bar      , Such that         | Indicates conditions on sets
    % ...           , Ellipses      , And so forth      | Elides numbers from the entire set for brevity
    \centering
    \caption{Set Roster Notation}
		\rowcolors{2}{}{gray!10}
    \begin{tabular}{ c l l l }
        \toprule
        Symbol  & Name          & Meaning           & Usage \\
        \midrule
        \{ \}   & Curly Braces  & The set of all    & Denotes a set \\
        |       & Pipe/Bar      & Such that         & Indicates conditions on sets \\
        ...     & Ellipses      & And so forth      & Elides numbers from the entire set for brevity \\
        \bottomrule
    \end{tabular}
    \label{tab:tbl-set-notation}
\end{table}

\subsection{Common Number Sets}
% TODO: Add sets for the groups of numbers
% TODO: Develop set notation functions to make writing easier
% TODO: Fix table headers not being centered

% TODO: Wrap the text on this
\begin{table}[!htbp]
    % Symbol        ,   Name        , Domain
    % Z             , Integers      , 
    % Q             , Rationals     , 
    % R             , Reals         , 
    \centering
		\rowcolors{2}{}{gray!10}
    \begin{tabular}{ c l l }
        \toprule
        Symbol  & Name      & Set \\
        \midrule
        \Z      & Integers  & $\{-3,-2,-1,0,1,2,3, ... \}$ \\
        \Q      & Rationals & $\{ -1/2, 0.333, ... \}$ \\
        \R      & Reals     & $\{ \sqrt{5} \}$ \\
        \bottomrule
    \end{tabular}
    \label{tab:tbl-number-sets}
    \caption{Common Number Sets}
\end{table}

\starON
**Note:** The $^+$ superscript denotes positive number elements in these sets. 

**Note:** The $^{nonneg}$ superscript denotes non-negative numbers (0, 1, 2, 3, ...). 
\starOFF

\subsection{Set Theory Notation}
\begin{table}[!htbp]
    \centering
    \caption{Set Theory Notation}
    \rowcolors{2}{}{gray!10}
    \begin{tabular}{ c l l l }
        \toprule
        Symbol      & LaTeX             & Name              & Meaning \\
        \midrule
        $\in$       & \verb|\in|        & In                & Is an element of ... \\
        $\notin$    & \verb|\notin|     & Not in            & Is not an element of ... \\
        $\subset$   & \verb|\subset|    & Proper Subset     & Is a proper subset of ... \\
        $\subseteq$ & \verb|\subseteq|  & Subset            & Is a subset of ... \\
        $\subset$   & \verb|\subset|    & Proper Superset   & Is a proper superset of ... \\
        $\subseteq$ & \verb|\subseteq|  & Superset          & Is a superset of ... \\
        $\cup$      & \verb|\cup|       & Set Union         & The set of elements in either set A or B \\
        $\cap$      & \verb|\cap|       & Set Intersection  & The set of elements in both set A and B \\
        $\setminus$ & \verb|\setminus|  & Set Difference    & The set of elements of A that are not in B \\
        \bottomrule
    \end{tabular}
    \label{tab:tbl-set-theory-notation}
\end{table}

\newpage

\begin{definition}[Subset]\label{def:subset-txt}
If A, B are sets, then A is a subset of B if and only if every element of A is an element of B
\end{definition}

% TODO Find symbols for this shit
%\begin{definition}[Subset]\label{def:subset}
    %Let A, B be sets,
        %A subset B iff for all elem a in A, a is in B
%\end{definition}

% TODO: Add symbolic variant definitions for sets, distinguish with text variants with -txt

\starON
**Note:** Contained-in is an alternative way of saying 'is a subset of'.
\starOFF

\begin{definition}[Proper Subset]\label{def:proper-subset-txt}
Let A, B be sets, A is a proper subset of B if and only if every element of A is in B, but there
is at least one element of B not in A.
\end{definition}

\subsection{Tuples}

\begin{definition}[Ordered Pair]\label{def:ordered-pair}
    A pair of elements $(a, b)$ where a is the first
    element of the pair, and b is the second. 
\end{definition}

\begin{definition}[Ordered Pair Equality]\label{def:ordered-pair-eq}
    Two ordered pairs $(a,b)$, $(c,d)$ are equal if and only if
    $a = c$ and $b = d$. In other words:

    \begin{equation}
        (a,b) = (c,d) \iff (a = c) \land (b = d)
    \end{equation}
\end{definition}

\begin{definition}[Tuples]\label{def:tuple}
    Let $n$ be a positive integer and let $x_1, x_2, ... , x_n$ be elements.

    An ordered $n$-tuple $(x_1, x_2, ... , x_n)$ is sorted in ascending order
    starting with $x_1$ to $x_n$. An ordered 2-tuple is an ordered pair,
    and an ordered 3-tuple is an ordered triple.
\end{definition}

\begin{definition}[Tuple Equality]\label{def:tuple-eq}
    Two ordered n-tuples $(x_1, x_2, ..., x_n)$, $(y_1, y_2, ..., y_n)$
    are equal if and only if:

    \begin{equation}
        (x_1, x_2, ..., x_n) = (y_1, y_2, ..., y_n) \iff x_1 = y_1, ... x_n = y_n
    \end{equation}
\end{definition}

In other words, all elements of one tuple must be equal to all the elements of the other tuple.

\begin{definition}[Cartesian Product]\label{def:cartesian-product}
    Given sets $A_1, A_2, ..., A_n$, the Cartesian product of $A_1, A_2, ..., A_n$
    denoted

    $A_1 \times A_2 \times ... \times A_n$ is the set of all ordered $n$-tuples
    $(a_1, a_2, ..., a_n)$ where $a_1 \in A_1, a_2 \in A_2, ... a_n \in A_n$
    \begin{equation}
        \begin{aligned}
            A_1 \times A_2 \times ... \times A_n &= \{ (a_1, a_2, ..., a_n) | a_1 \in A_1, a_2 \in A_2, ... a_n \in A_n \} \\
            A_1 \times A_2 &= \{ (a_1, a_2) | a_1 \in A_1, a_2 \in A_2 \}
        \end{aligned}
    \end{equation}
\end{definition}

\begin{definition}[Strings]\label{def:string}
    Let $n$ be a positive integer. Given a finite set $A$, a string
    of length $\frac{n}{A}$ is an ordered $n$-tuple of elements of $A$.
    The elements of $A$ are called characters. The null string of $A$
    denoted $\lambda$ is a string with no characters and length of $0$.
    If $A = \{0, 1\}$ then a string over $A$ is a bit string.
\end{definition}

\newpage
\section{Predicate Logic}

\subsection{Logical Operators}
\begin{table}[!htbp]
    % Logical Symbol    , Programming Symbol    , LaTeX Symbol  , Name          , Meaning
    % &&                , &&                    , \wedge        , Logical And   , Conjunction of two statements
    % ||                , ||                    , \lor          , Logical Or    , Disjunction of two statements
    % Not               , !                     , \lnot         , Negation      , Negates a statement
    % Xor               , xor                   , \xor          , Exclusive Or  , Logical Or, but both statements cannot be true
    \centering
    \caption{Logical Operators}
		\rowcolors{2}{}{gray!10}
    \begin{tabular}{ c c l l l l  }
        \toprule
        Symbol      & Programming   & LaTeX         & Name          & Meaning \\
        \midrule
        $\land$     & \&\&          & \verb|\wedge| & And           & Conjunction of two terms \\
        $\lor$      & ||            & \verb|\lor|   & Or            & Disjunction of two terms \\
        $\lxor$     & \string^      & \verb|\lxor|  & Xor           & Exclusive logical or \\
        $\neg$      & !             & \verb|\neg|   & Not           & Negation of a term \\
        $\to$       &               & \verb|\to|    & Implies       & Logical implication \\
        $\iff$      &               & \verb|\iff|   & Biconditional & Logical Equivalence \\
        \bottomrule
    \end{tabular}
    \label{tab:tbl-logic-ops}
\end{table}

\starON
**Note:** An alternative syntax for negation is tilde: $\sim$ (\verb|\sim|).

**Note:** An alternative syntax for logical equivalence is $\equiv$ (\verb|\equiv|).

**Note:** In the biconditional, the hypothesis implies the conclusion and vice versa.

**Note:** Exclusive or contains almost the same truth values as logical or, with the exception
that both operands of the connective cannot both be simultaneously true.
\starOFF

\subsubsection{Truth Values}

$T$ is used to denote a true value and $F$ is used to denote a false value.

\begin{definition}[Negation]\label{def:negation}
    The interchanging of truth values in a statement.
    Given statement p, the resulting statement is "not p"
\end{definition}

\begin{definition}[Conjunction]\label{def:conjunction}
    A compound statement read as "p and q" $p \land q$ that 
    is true only when both p and q are true, and false otherwise.
\end{definition}

\begin{definition}[Disjunction]\label{def:disjunction}
    A compound statement read as "p or q" $p \lor q$
    that is false when both p and q are false, and true otherwise.
\end{definition}

\begin{definition}[Statement Form]\label{def:statement-form}
    An expression of statement variables
    and logical connectives that becomes a proper statement
    when actual statements are substituted in for the variables.
\end{definition}

\begin{definition}[Truth Table]\label{def:truth-table}
    The enumeration of all possible truth values for a given statement.
\end{definition}

\begin{definition}[Logical Equivalence]\label{def:logical-eq}
    Two statements are logically equivalent (denoted $\iff$),
    if and only if they have identical truth values
    for every possible substitution for the statement variables. 
\end{definition}

\large{Testing Statements $P$, $Q$ for Logical Equivalence}
\begin{enumerate}
    \item Create the truth tables for $P$, $Q$
    \item For all truth values, check that $P$'s values match $Q$'s
\end{enumerate}

If for every row of $P$, the truth value is the same as $Q$, then 
the two forms are logically equivalent. Otherwise if even one
value is different, the two are not.

\begin{definition}[Tautology]\label{def:tautology}
A statement form that holds true for all possible statement substitutions regardless of the individual statements truth values.
\end{definition}

\begin{definition}[Contradiction]\label{def:contradiction}
A statement form that holds false for all possible statement substitutions regardless of the individual statements truth values.
\end{definition}

\starON
**Note:** $T$ or $t$ denotes a tautology

**Note:** $F$ or $c$ denotes a contradiction

The usage depends on whichever variant is used. In our case we will use
$t$ and $c$.
\starOFF

\subsection{Laws of Logic}

\begin{table}[!htbp]
    \centering
    \rowcolors{2}{}{gray!10}
    \begin{tabular}{ l l l }
        \toprule
        Logical Law & Form & Additional Form \\
        \midrule
        Commutative Law     & $p \lor q \iff q \lor p$
                            & $p \land q \iff q \land p$ \\
        Associative Law     & $p \lor (q \lor r) \iff (p \lor q) \lor r$
                            & $p \land (q \land r) \iff (p \land q) \land r$ \\
        Distributive Law    & $p \lor (q \land r) \iff (p \lor q) \land (p \lor r)$ 
                            & $p \land (q \lor r) \iff (p \land q) \lor (p \land r)$ \\
        Inverse Law         & $p \lor \neg p \iff t$
                            & $p \land \neg p \iff c$ \\
        Identity Law        & $p \lor c \iff p$
                            & $p \land t \iff p$ \\
        Double Negation Law & $\neg \neg p \iff p$ & \\
        Idempotent Law      & $p \lor p \iff p$ 
                            & $p \land p \iff p$ \\
        Domination Law      & $p \lor t \iff t$
                            & $p \land c \iff c$ \\
        DeMorgan's Laws     & $\neg(p \lor q) \iff \neg p \land \neg q$
                            & $\neg(p \land q) \iff \neg p \lor \neg q$ \\
        Absorption Laws     & $p \lor (p \land q) \iff p$
                            & $p \land (p \lor q) \iff p$ \\
        \bottomrule
    \end{tabular}
    \label{tab:tbl-laws-of-logic}
    \caption{Laws of Logic. These statements hold true for any statements $p$, $q$, $r$, any tautology $t$, and contradiction $c$.}
\end{table}

\starON
**Note:** The Inverse Law is also known as the Negation Law.

**Note:** The Domination Law is also known as the Universal Bound Law.
\starOFF

\subsection{Conditional Statements}

% TODO: Truth table for p -> q
\begin{definition}[Conditional]\label{def:conditional}
    If $p$, $q$ are statement variables, the conditional of $q$ by $p$
    is a statement that is false when $p$ is true, and $q$ is false, and true
    otherwise. $p$ is known as the hypothesis and $q$ is the conclusion.
\end{definition}

Conditional statements (also known as implications) are read as
"if $p$ then $q$" or "$p$ implies $q$".

Conditional statements are vacuously true (true by default).

\subsection{Variants of Implication}

\begin{definition}[Contrapositive]\label{def:contrapositive}
    The contrapositive of conditional statement: "if $p$ then $q$",
    is "if $\neg q$ then $\neg p$". Symbolically:
    \begin{equation}
        p \to q \iff \neg q \to \neg p
    \end{equation}
\end{definition}

\begin{definition}[Converse]\label{def:converse}
    Given conditional statement: "if $p$ then $q$"
    The converse is "if $q$ then $p$":
\end{definition}

\begin{table}[!htbp]
    \centering
    \rowcolors{2}{}{gray!10}
    \begin{tabular}{ l c c}
        \toprule
        Conditional Statement & Form                & $\iff$ $p \to q$ \\
        \midrule
        Implication         & $p \to q$             & Yes \\
        Contrapositive      & $\neg q \to \neg p$   & Yes \\
        Converse            & $q \to p$             & No \\
        Inverse             & $\neg p \to \neg q$   & No \\
        \bottomrule
    \end{tabular}
    \label{tab:tbl-implication-variants}
    \caption{Variants of Conditional Statements}
\end{table}

\starON
**Note:** Although the converse and inverse are not logically equivalent to "if $p$ then $q$"
they are logically equivalent to each other.
\starOFF

\begin{definition}[Only If]\label{def:only-if}
    For statements $p$, $q$,
    $p$ only if $q$ means "if not $q$ then not $p$".
    or equivalently "if $p$ then $q$".
\end{definition}

\begin{definition}[Biconditional]\label{def:biconditional}
    The biconditional of statements $p$, $q$ is
    "$p$ if and only if $q$" denoted $p \iff q$. It
    is true if both $p$ and $q$ share the same truth values
    and false otherwise.
\end{definition}

\subsection{Logical Order of Operations}
\begin{enumerate}
    \item $\neg$
    \item $\land$, $\lor$ (Parenthesis may be needed).
    \item $\to$, $\iff$ (Parenthesis may be needed).
\end{enumerate}

\begin{definition}[Sufficient Condition]\label{def:sufficient-condition}
    For statements $r$, $s$,
    $r$ is a sufficient condition for $s$ implies "if $r$ then $s$".
\end{definition}

\begin{definition}[Necessary Condition]\label{def:necessary-condition}
    For statements $r$, $s$,
    $r$ is a necessary condition for $s$ implies "if not $r$ then not $s$".
\end{definition}

\subsection{Arguments}

% TODO: Arguments definition: Page: 66

\subsection{Rules of Inference}
\begin{table}
    \caption{Rules of Inference}
    \adjustbox{width=\textwidth}{
    \begin{tabular}[f]{ *{4}{cc} }
        \toprule
        Inference Rule & Argument Form & Alternate Form  & Inference Rule & Argument Form & Alternate Form \\
        \midrule
        Modus Ponens    &
                        \begin{argument}
                            \premise{p \to q}
                            \premise{p}
                            \conclusion{q}
                        \end{argument}
                        & &
        Elimination     & 
                        \begin{argument}
                            \premise{p \lor q}
                            \premise{\neg q}
                            \conclusion{p}
                        \end{argument}
                        &
                        \begin{argument}
                            \premise{p \lor q}
                            \premise{\neg p}
                            \conclusion{q}
                        \end{argument} \\
        \midrule
        Modus Tollens   &
                        \begin{argument}
                            \premise{p \to q}
                            \premise{\neg q}
                            \conclusion{\neg p}
                        \end{argument}
                        & &
        Transitivity    &
                        \begin{argument}
                            \premise{p \to q}
                            \premise{q \to r}
                            \conclusion{p \to r}
                        \end{argument}
                        & \\
        \midrule
        Generalization  &
                        \begin{argument}
                            \premise{p}
                            \conclusion{p \lor q}
                        \end{argument}
                        &
                        \begin{argument}
                            \premise{q}
                            \conclusion{p \lor q}
                        \end{argument}
                        &
        Proof By
        Division into Cases
                        &
                        \begin{argument}
                            \premise{p \lor q}
                            \premise{p \to r}
                            \premise{q \to r}
                            \conclusion{r}
                        \end{argument}
                        & \\
        \midrule
        Specialization  &
                        \begin{argument}
                            \premise{p \land q}
                            \conclusion{p}
                        \end{argument}
                        &
                        \begin{argument}
                            \premise{p \land q}
                            \conclusion{q}
                        \end{argument}
                        &
        Contradiction
        Rule
                        &
                        \begin{argument}
                            \premise{\neg p \to c}
                            \conclusion{p}
                        \end{argument}
                        & \\
        \midrule
        Conjunction     &
                        \begin{argument}
                            \premise{p}
                            \premise{q}
                            \conclusion{p \land q}
                        \end{argument}
                        & \\
        \bottomrule
    \end{tabular}
    }
    \label{tab:tbl-inference-rules}
\end{table}

% TODO: Entire section of gotchas

\subsection{Quantifiers}

\begin{table}[!htbp]
    % Table of quantifiers
    % Symbol    , Name          , Meaning
    % ∀         , Universal     , For all elements of the set
    % E         , Existential   , There exists an element of the set
    \centering
    \begin{tabular}{ c c l l  }
        \toprule
        Symbol      & LaTeX             & Name                      & Meaning \\
        \midrule
        $\exists$   & \verb|\exists|    & Existential Quantifier    & There exists at least one \\
        $\forall$   & \verb|\forall|    & Universal Quantifier      & For all \\
        \bottomrule
    \end{tabular}
    \label{tab:tbl-quantifiers}
    \caption{Quantifiers}
\end{table}

% TODO: Truth Set
\begin{definition}[Truth Set]\label{def:truth-set}
    For a predicate $P(x)$, and domain $D$, the set of all elements of $D$ 
    for which $P(x)$ holds true is known as the truth set.
    \begin{equation*}
        \{ x \in D \mid P(x) \}
    \end{equation*}
\end{definition}

\begin{definition}[Universal Quantified Statements]\label{def:universal-quantified-stmts}
    For a predicate $Q(x)$, domain $D$, a universal statement is of the form:
    \begin{equation*}
        \forall x \in D \mid Q(x)
    \end{equation*}

    This statement is true if and only if the predicate $Q(x)$ is true for every element of $D$.
    If even one counterexample is found, then the statement is false.
\end{definition}

\begin{definition}[Existential Quantified Statements]\label{def:existential-quantified-stmts}
    For a predicate $Q(x)$, domain $D$, an existential statement is of the form:
    \begin{equation*}
        \exists x \in D \mid Q(x)
    \end{equation*}

    This statement is true if and only if there exists a single element for which the 
    predicate $Q(x)$ is true. It is false if no elements exist that uphold the predicate.
\end{definition}

Variable binding is the restriction of a variable to its quantifier.

Variable scope refers to the lifetime of a given variable.

\begin{definition}[Negation of Universal Statements]\label{def:neg-universal-stmts}
    \begin{equation*}
        \lsim (\forall x \in D, Q(x)) \equiv \exists x \in D \text{ such that } \lsim Q(x)
    \end{equation*}
\end{definition}

\begin{definition}[Negation of Existential Statements]\label{def:neg-existential-stmts}
    \begin{equation*}
        \lsim (\exists x \in D, Q(x)) \equiv \forall x \in D \text{ such that } \lsim Q(x)
    \end{equation*}
\end{definition}

\begin{definition}[Universal Instantiation]\label{def:universal-instantiation}
    If a property is true for all elements of a set, then it is true of any element taken from the set.
\end{definition}

\begin{bbox}{\centering Universal Modus Ponens}
    \begin{center}
        \begin{argument}
        \premise{\forall x, \text{ if } P(x) \text{ then } $Q(x)$}
        \premise{$P(a)$ \text{ for a particular } $a$}
        \conclusion{$Q(a)$}
    \end{argument}
    \end{center}
\end{bbox}

\begin{bbox}{\centering Universal Modus Tollens}
    \begin{center}
        \begin{argument}
        \premise{\forall x, \text{ if } P(x) \text{ then } $Q(x)$}
        \premise{\lsim Q(a) \text{ for a particular } a}
        \conclusion{\lsim P(a)}
    \end{argument}
    \end{center}
\end{bbox}

\end{document}
