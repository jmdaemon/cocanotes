\documentclass{article}
\usepackage[table]{xcolor}
\usepackage[noparindent]{coco}
\usepackage[numbersets]{shortmath}
\usepackage{thmstyles}

\geometry{
    left=2cm,
    top=2cm,
    right=2cm,
    bottom=2cm,
}

\begin{document}
% Cover Page
\begin{titlepage}
    \begin{center}
        \Huge Calculus
    \end{center}
\end{titlepage}

\newpage
\tableofcontents

\section{Calculus}

\section{Fundamentals}

% TODO: Inequalities with Abs 
% TODO: Midpoint of an interval
% TODO: Factoring Review
% TODO: Quadratic Formula, Special Products (Page 19), Binomial Theorem, Grouping,
% TODO: Synthetic Division, Long Division, Rational Zero Theorem5
% TODO: Finding intercepts

% TODO: Distance Formula, Pythagorean Theorem, Midpoint Formula,
% TODO: Standard form of the equation of a circle (page 47)
% TODO: Completing the square (Page 48)
% TODO: Finding points of intersection (Page 49)

\section{Formulas} %

\begin{equation*}
    
\end{equation*}

\section{Business}

Supply equation: The relationship of a quantity supplied $x$ to the unit price $p$ of a product.

Demand equation: The relationship of a quantity demanded $x$ to the unit price $p$ of a product.

Cost function: The total cost produced for $x$ many products.

Revenue function: The total revenue generated for selling $x$ many products.

Profit function: The total profit gained for selling $x$ many products

Break-even point: The equilibrium point where $R = C$

\subsection{Marginals}

% TODO: Linear equations, slope-intercept form, standard form, point-slope form
% TODO: Slope formula of two points, slope for parallel, perpendicular lines
% TODO: Equations of lines (Page 62)

% TODO: Definition of a function, independent, dependent variables, domain, range
% TODO: Determining functions, vertical line test (Page 71)
% TODO: Function composition (Page 74)

% TODO: Definition of one-to-one, horizontal line test (Page 71)
% TODO: Definition of inverse function, how to determine inverse functions (Page 75)

\section{Limits}

\begin{definition}[Definition of a Limit]\label{def:limit}
    If $f(x)$ becomes arbitrarily close to a number $L$ as $x$ approaches $c$ from either side then
    \begin{equation*}
        \lim_{x \to c} F(x) = L
    \end{equation*}
    The limit of $f(x)$ as $x$ approaches $c$ is $L$.
\end{definition}

\subsection{Limit Properties}
% TODO: Properties of Limits (Page 84-85)
% Make boxed environments
% Properties of Limits

\subsection{Limit Operations}

\begin{definition}[Existence of a Limit]\label{def:limit-exists}
    If $f$ is a function, and $c$, $L$ are real numbers, then
    \begin{equation*}
        \lim_{x \to c} f(x) = L
    \end{equation*}
    if and only if
    \begin{equation*}
        \lim_{x \to c^-} f(x) = \lim_{x \to c^+} f(x) = L
    \end{equation*}
\end{definition}

\begin{definition}[Definition of Continuity]\label{def:continuity}
    Let $c$ be a number in the interval $(a,b)$, and let $f$ be a function whose domain
    contains the interval $(a,b)$. The function $f$ is continuous at the point $c$ if:
    \begin{enumerate}
        \item $f(c)$ is defined.
        \item $\lim_{x \to c} f(x)$ exists
        \item $\lim_{x \to c} f(x) = f(x) = L$
    \end{enumerate}
    If $f$ is continuous at every point in the interval $(a,b)$ then it is continuous on
    an open interval $(a,b)$.
\end{definition}

\subsection{Continuity for Various Functions}

% TODO: Continuity of polynomial, rational functions (Page 95)
% Table for continuity for various functions

% Function Name | Symbol | Domain | Range
% Rational Function | P(X) / Q(X) | x \neq 0 | \R
% Polynomial Functions | ax^n + bx^n-1 + ... | x \in \R | x \in \R

\begin{definition}[Continuity on a Closed Interval]\label{def:continuity-closed-int}
    Let $f$ be defined on a closed interval $[a,b]$.
    If $f$ is continuous on the open interval $(a,b)$ and
    \begin{equation*}
        \begin{aligned}
            \lim_{x \to a^+} f(a) & \text{ and } & \lim_{x \to b^-} f(b)
        \end{aligned}
    \end{equation*}
    then $f$ is continous on the closed interval $[a,b]$,
    as well as the left and right endpoints.
\end{definition}

% TODO: Step function, floor, ceiling functions (Page 98)

% TODO: Formula for simple and compound interests

% TODO: Tangent lines, secant lines (Page 115, 117)

\section{Theorems}

\subsection{Intermediate Value Theorem}


\subsection{Squeeze Theorem}

\section{Derivatives}

\begin{definition}[Definition of a Derivative]\label{def:derivative}
    The derivative of $f$ at $x$ is:
    \begin{equation*}
        f'(x) = \lim_{\Delta x \to 0} \frac{f(x + \Delta x) - f(x)}{\Delta x}
    \end{equation*}
    given that the limit exists and the function is differentiable at $x$.
\end{definition}

Derivatives do not exist at:

\begin{itemize}
    \item Vertical lines/tangents: The reason is often division by zero, but additionally the curve is too sharp.
    \item Discontinuities: The limit does not exist.
    \item Cusps: The function is too sharp.
    \item Nodes: The function changes sign too quickly/cannot differentiate with just one point.
\end{itemize}

Note that differentiability implies continuity. If a function
$f$ is differentiable at $x = c$ then $f$ is continuous at $x = c$.

\subsection{Differentiation Rules}

% TODO: Differentiation Rules (Pages: 126, 127, 129, 132) 

\end{document}
