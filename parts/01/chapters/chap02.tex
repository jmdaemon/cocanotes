\documentclass[../../../main.tex]{subfiles}
\graphicspath{{\subfix{./figures/}}}

\begin{document}

\chapter{Fundamentals of Logic}
\label{chap:logic}

\section{Basic Connectives and Truth Tables}

\subsection{Definition of a Proposition}

\begin{definition}[Propositions]
Propositions are declarative statements that are true or false but not both
\end{definition}
    
\begin{definition}[Primitives]
Primitive statements are propositions that can't be simplified or broken down further.
\end{definition}

New propositions can be formed by
\begin{itemize}
    \item Negating a statement
    \item Combining statements with logical connectives.
\end{itemize}

\subsection{Logical Connectives}
\begin{itemize}
    \item Conjunction ($\land$) $p \land q$
    \item Disjunction ($\lor$) $p \lor q$
        \begin{itemize}
            \item Logical OR is true when both/either propositions p,q are true.
            \item Exclusive OR is true when exactly p or q is true, but not both.
        \end{itemize}
    \item Implication ($\implies$) $p \implies q$
    \item Means:
        \begin{enumerate}[label=\roman*)]
            \item if p then q
            \item p is sufficient for q
            \item p is a sufficient condition of q
            \item q is necessary for p
            \item q is a necessary condition for p
            \item p only if q
        \end{enumerate}
    \item where:
        \begin{tabular}{@{}l}
            p: hypothesis \\
            q: conclusion/implication
        \end{tabular}
    \item Biconditional [spaceship operator] ($\iff$) $p \iff q$
    \item Means:
        \begin{itemize}
            \item p is necessary and sufficient for q
            \item Abbreviated as iff meaning if and only if
        \end{itemize}
\end{itemize}

Note that the statement: "x is an integer"
is not a proposition because its truth value
is indeterminable until a value is assigned 
to x.

\subsection{Truth Tables}

Let

%\begin{tabular}{@{}l}
    %0: True \\
    %1: False
%\end{tabular}
\begin{tabularx}{0.80\linewidth}{@{}lX}
    & 0: True \\
    & 1: False
\end{tabularx}

\begin{table}[!htb]
    \caption{Truth Tables for Logical Connectives}
    \centering
    % Negation
    \begin{subtable}{0.15\linewidth}
        \centering
        \caption{Negation}
        \label{tbl:logical-negation}
        $\begin{array}{ *{2}{c} }
            % 2 centered columns
            \toprule \\
            p & \neg q \\
            \midrule \\
            0 & 1 \\
            1 & 0 \\
            \bottomrule
        \end{array}$
    \end{subtable}%
    % Other Logical Connectives
    \begin{subtable}{0.5\linewidth}
        \centering
        \caption{Other Logical Connectives}
        \label{tbl:logic-connectives}
        $\begin{array}{ *{7}{c} }
            % 7 centered columns
            \toprule \\
            p & q & p \land q & p \lor q & p \lxor q & p \implies q & p \iff q \\
            \midrule \\
            0 & 0 & 0 & 0 & 0 & 1 & 1 \\
            0 & 1 & 0 & 1 & 1 & 1 & 0 \\
            1 & 0 & 0 & 1 & 1 & 0 & 0 \\
            1 & 1 & 1 & 1 & 0 & 1 & 1 \\
            \bottomrule
        \end{array}$
    \end{subtable}
\end{table}

Note:
\begin{itemize}
    \item $p \land q$ is only true when p,q are true
    \item $p \lor q$ is only false when p,q are false
    \item $p \lxor q$ is only true when exactly one of p,q is true
    \item $p \implies q$ is true in all cases except when p is true and q is false
    \item $p \iff q$ is true only whe p,q have the same truth value
\end{itemize}

\end{document}
