%\documentclass[../../../../main.tex]{subfiles}
%\graphicspath{{\subfix{../../../../figures/}, ./figures}}

%\begin{document}
\section{Logical Equivalence: The Laws of Logic}

In all areas of maths it is vital to understand whether the
entities we are studying are equal or essentially the same.

\newcommand\negporq{\neg p \lor q}
\newcommand\ptoq{p \to q}

For example take these two statements $\negporq$
and $\ptoq$

\begin{table}
    \centering
    \caption*{Truth Table for Equivalent Statements}
    \label{tbl:equivalent-statements}
    \begin{subtable}{\linewidth}
        \centering
        $\begin{array}{ *{5}{c} }
            % 5 centered columns
            \toprule \\
            p & q & \neg p & \negporq & \ptoq \\
            \midrule \\
            0 & 0 & 1 & 1 & 1 \\
            0 & 1 & 1 & 1 & 1 \\
            1 & 0 & 0 & 0 & 0 \\
            1 & 1 & 0 & 1 & 1 \\
            \bottomrule
        \end{array}$
    \end{subtable}%
\end{table}

Here statements $\negporq$ and $\ptoq$
are said to be logically equivalent.

\begin{definition}[Logical Equivalence]
    Two statements $s_1$, $s_2$ are \underline{logically equivalent} when $s_1$ is true
    exactly when $s_2$ is true, and false exactly when $s_2$ is false.
\end{definition}

We use $\iff$ to denote logical equivalency.

When $s_1 \iff s_2$ we can represent $s_1$ using the same truth table as $s_2$,
since they both have the same truth values.

Thus we can represent the connective implication in terms of negation and disjunction.

\begin{table}[!htb]
    \centering
    \caption*{Logically equivalent truth table for implication}
    \label{tbl:log-eql-bicond}
    \begin{subtable}{\linewidth}
        \centering
        $\begin{array}{ *{6}{c} }
            % 6 centered columns
            \toprule \\
            p & q & p \to q & q \to p & (p \to q) \land (q \to p) & p \iff q \\
            \midrule \\
            0 & 0 & 1 & 1 & 1 & 1 \\
            0 & 1 & 1 & 0 & 0 & 0 \\
            1 & 0 & 0 & 1 & 0 & 0 \\
            1 & 1 & 1 & 1 & 1 & 1 \\
            \bottomrule
        \end{array}$
    \end{subtable}%
\end{table}

We can also express the exclusive or connective in terms of conjunctions and disjunctions.

\begin{table}[!htb]
    \centering
    \caption*{Logically equivalent truth table for exclusive or}
    \label{tbl:log-eql-xor}
    \begin{subtable}{\linewidth}
        \centering
        $\begin{array}{ *{7}{c} }
            % 7 centered columns
            \toprule \\
            p & q & p \lxor q & p \lor q & p \land q & \neg (p \land q) & (p \lor q) \land \neg (p \land q) \\
            \midrule \\
            0 & 0 & 0 & 0 & 0 & 1 & 0 \\
            0 & 1 & 1 & 1 & 0 & 1 & 1 \\
            1 & 0 & 1 & 1 & 0 & 1 & 1 \\
            1 & 1 & 0 & 1 & 1 & 0 & 0 \\
            \bottomrule
        \end{array}$
    \end{subtable}%
\end{table}

\newpage
\subsection{The Laws of Logic}

% Requirements
% A caption for the table
% Pairs of vertically aligned statements
% Two columns
\begin{table}[!htb]
    \caption*{For any primitive statements $p, q, r$, any tautology $\taut$, and any contradiction $\contra$}
    \centering
    \begin{subtable}{\linewidth}
        \centering
        \label{tbl:laws-of-logic}
        $\begin{array}{ *{2}{l} }
            \toprule \\
            \neg \neg p \iff p & \text{Double Negation} \\
            \\
            \neg(p \lor q) \iff \neg p \land \neg q & \text{DeMorgan's Laws} \\
            \neg(p \land q) \iff \neg p \lor \neg q & \\
            \\
            p \lor q \iff q \lor p & \text{Commutative Laws} \\
            p \land q \iff q \land p & \\
            \\
            p \lor (q \lor r) \iff (p \lor q) \lor r & \text{Associative Laws} \\ 
            p \land (q \land r) \iff (p \land q) \land r & \\
            \\
            p \lor (q \land r) \iff (p \lor q) \land (p \lor r) & \text {Distributive Laws} \\
            p \land (q \lor r) \iff (p \land q) \lor (p \land r) & \\
            \\
            p \lor p \iff p & \text {Idempotent Laws} \\
            p \land p \iff p & \\
            \\
            p \lor \contra \iff p & \text{Identity Laws} \\
            p \land \taut \iff p & \\
            \\
            p \lor \neg p \iff \taut & \text{Inverse Laws} \\
            p \land \neg p \iff \contra & \\
            \\
            p \lor \taut \iff \taut & \text {Domination Laws} \\
            p \land \neg p \iff \contra & \\
            \\
            p \lor (p \land q) \iff p & \text{Absorption Laws} \\
            p \land (p \lor q) \iff p & \\
            %\multirow{2}{*}{} & \neg \neg p \iff p & \text{} & \text{Double Negation} \\
            %\multirow{2}{*}{} & p \lor q \iff q \lor p & p \land q  \ifq \land p & \text{Commutative Laws} \\
            \bottomrule
        \end{array}$
        %\begin{tabular}{ *{2}{c} }
            %% 2 centered columns
            %$\neg \neg p \iff p$ & Law of Double Negation \\
            %\begin{tabularx}{0.80\linewidth}{@{}lX}
                %& \neg (p \lor \\
            %\end{tabularx} \\
        %\end{tabular}
    \end{subtable}%
\end{table}

Note that you may choose to prove these laws by comparing their truth tables with each other.
These truth tables are not shown, instead left as an exercise for the reader.

Once you have convinced yourselves of the truth of these laws, then
the below definition gives us a method to find the dual of a statement.

\begin{definition}[Dual Statements]\label{def:dual-statements}
    For any statement $s$ with no component connectives other than $\land, \lor$,
    then the dual of $s$, $s^d$ is obtained by replacing every $\land$ with $\lor$,
    every $\lor$ with $\land$, every $\taut$ with $\contra$ and every $\contra$ with $\taut$.
\end{definition}

Note that for a primitive statement $p$ the dual of $p$, $p^d$ is the same primitive statement.

\begin{theorem}[The Principle of Duality]\label{thm:principle-of-duality}
    Let $s, t$ be statements consisting of no logical connectives other than $\lor$ and $\land$.
    If $s \iff t$, then $s^d \iff t^d$.
\end{theorem}

We will use the definition of a \cref{def:dual-statements} and the \cref{thm:principle-of-duality},
to rewrite compound statements in terms of $\land$ and $\lor$, to find their dual statements.

\subsection{Substitution Rules}

\begin{enumerate}
    \item Suppose a compound statement $P$ is a tautology. If
        $p$ is a primitive statement appearing in $P$, and we
        substitute every occurence of p by statement q,
        then the resulting statement $P_1$ is also a tautology.
    \item Let $q, p$ be statements that appear in compound statement $P$
        such that $q \iff p$. If $p$ is replaced by one or more occurrences
        of $q$, in P, then the resulting statement $P_1 \iff P$.
\end{enumerate}

\subsection{Variants of Implication}

For a given implication $p \to q$ there exists:

\begin{itemize}
    \item The implication ($p \to q$)
    \item The contrapositive ($\neg q \to \neg p$)
    \item The converse ($q \to p$) \item The inverse $\neg p \to \neg q$ \end{itemize}

Note:

\begin{itemize}
    \item If p is true and q is false, then the implication and contrapositive are false,
        while the converse and inverse are true.
    \item If p is false and q is true, then the implication and contrapositive are true,
        while the converse and inverse are false.
    \item If p and q are both true or false, then the implication, contrapositive, converse and inverse
        are all true.
\end{itemize}

or more succintly, for an implication $p \to q$:

\begin{table}[!htb]
    \begin{subtable}{\linewidth}
        \centering
        \caption*{Truth values of implication variants}
        \label{tbl:implication-variants}
        \begin{tabular}{ *{2}{l} }
            % 2 centered columns
            \toprule \\
            (p, q) & result \\
            \midrule \\
            true, true & All statement variants are true \\
            true, false & Implication \& contrapositive are false, converse \& inverse are true \\
            false, true & Implication \& contrapositive are true, converse \& inverse are false \\
            false, false & All statement variants are true \\
            \bottomrule
        \end{tabular}
    \end{subtable}%
\end{table}

%\end{document}
