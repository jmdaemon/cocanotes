\documentclass[../../../../main.tex]{subfiles}
\graphicspath{{\subfix{../../../../figures/}, ./figures}}

\begin{document}
\section{Logical Equivalence: The Laws of Logic}

In all areas of maths it is vital to understand whether the
entities we are studying are equal or essentially the same.

\newcommand\negporq{\neg p \lor q}
\newcommand\ptoq{p \to q}

For example take these two statements $\negporq$
and $\ptoq$

\begin{table}
    \centering
    \caption*{Truth Table for Equivalent Statements}
    \label{tbl:equivalent-statements}
    \begin{subtable}{\linewidth}
        \centering
        $\begin{array}{ *{5}{c} }
            % 5 centered columns
            \toprule \\
            p & q & \neg p & \negporq & \ptoq \\
            \midrule \\
            0 & 0 & 1 & 1 & 1 \\
            0 & 1 & 1 & 1 & 1 \\
            1 & 0 & 0 & 0 & 0 \\
            1 & 1 & 0 & 1 & 1 \\
            \bottomrule
        \end{array}$
    \end{subtable}%
\end{table}

Here statements $\negporq$ and $\ptoq$
are said to be logically equivalent.

\begin{definition}[Logical Equivalence]
    Two statements $s_1$, $s_2$ are \underline{logically equivalent} when $s_1$ is true
    exactly when $s_2$ is true, and false exactly when $s_2$ is false.
\end{definition}

We use $\iff$ to denote logical equivalency.

When $s_1 \iff s_2$ we can represent $s_1$ using the same truth table as $s_2$,
since they both have the same truth values.

Thus we can represent the connective implication in terms of negation and disjunction.

\begin{table}[!htb]
    \centering
    \caption*{Logically equivalent truth table for implication}
    \label{tbl:log-eql-bicond}
    \begin{subtable}{\linewidth}
        \centering
        $\begin{array}{ *{6}{c} }
            % 6 centered columns
            \toprule \\
            p & q & p \to q & q \to p & (p \to q) \land (q \to p) & p \iff q \\
            \midrule \\
            0 & 0 & 1 & 1 & 1 & 1 \\
            0 & 1 & 1 & 0 & 0 & 0 \\
            1 & 0 & 0 & 1 & 0 & 0 \\
            1 & 1 & 1 & 1 & 1 & 1 \\
            \bottomrule
        \end{array}$
    \end{subtable}%
\end{table}

We can also express the exclusive or connective in terms of conjunctions and disjunctions.

\begin{table}[!htb]
    \centering
    \caption*{Logically equivalent truth table for exclusive or}
    \label{tbl:log-eql-xor}
    \begin{subtable}{\linewidth}
        \centering
        $\begin{array}{ *{7}{c} }
            % 7 centered columns
            \toprule \\
            p & q & p \lxor q & p \lor q & p \land q & \neg (p \land q) & (p \lor q) \land \neg (p \land q) \\
            \midrule \\
            0 & 0 & 0 & 0 & 0 & 1 & 0 \\
            0 & 1 & 1 & 1 & 0 & 1 & 1 \\
            1 & 0 & 1 & 1 & 0 & 1 & 1 \\
            1 & 1 & 0 & 1 & 1 & 0 & 0 \\
            \bottomrule
        \end{array}$
    \end{subtable}%
\end{table}


\subsection{The Laws of Logic}

%\begin{tabularx}{0.80\linewidth}{@{}lX}
    %\begin{itemize}
        %\item asdf & asdf
    %\end{itemize}
%\end{tabularx}

\end{document}
