\documentclass[../../../../main.tex]{subfiles}
\graphicspath{{\subfix{../../../../figures/}, ./figures}}


\begin{document}
    
\section{Basic Connectives and Truth Tables}
\subsection{Definition of a Proposition}

\begin{definition}[Propositions]
Propositions are declarative statements that are true or false but not both
\end{definition}
    
\begin{definition}[Primitives]
Primitive statements are propositions that can't be simplified or broken down further.
\end{definition}

New propositions can be formed by
\begin{itemize}
    \item Negating a statement
    \item Combining statements with logical connectives.
\end{itemize}

\subsection{Logical Connectives}
\begin{itemize}
    \item Conjunction ($\land$) $p \land q$
    \item Disjunction ($\lor$) $p \lor q$
        \begin{itemize}
            \item Logical OR is true when both/either propositions p,q are true.
            \item Exclusive OR is true when exactly p or q is true, but not both.
        \end{itemize}
    \item Implication ($\implies$) $p \implies q$
    \item Means:
        \begin{enumerate}[label=\roman*)]
            \item if p then q
            \item p is sufficient for q
            \item p is a sufficient condition of q
            \item q is necessary for p
            \item q is a necessary condition for p
            \item p only if q
        \end{enumerate}
    \item where:
        \begin{tabular}{@{}l}
            p: hypothesis \\
            q: conclusion/implication
        \end{tabular}
    \item Biconditional [spaceship operator] ($\iff$) $p \iff q$
    \item Means:
        \begin{itemize}
            \item p is necessary and sufficient for q
            \item Abbreviated as iff meaning if and only if
        \end{itemize}
\end{itemize}

Note that the statement: "x is an integer"
is not a proposition because its truth value
is indeterminable until a value is assigned 
to x.

\subsection{Truth Tables}

Let

%\begin{tabular}{@{}l}
    %0: True \\
    %1: False
%\end{tabular}
\begin{tabularx}{0.80\linewidth}{@{}lX}
    & 0: False \\
    & 1: True
\end{tabularx}

% TODO Add to margins
Note that on POSIX systems, 0 is used to mean "no error" or any Truthy value
with any non zero value denoting an error or Falsy value.

\begin{table}[!htb]
    \caption{Truth Tables for Logical Connectives}
    \centering
    % Negation
    \begin{subtable}{0.15\linewidth}
        \centering
        \caption{Negation}
        \label{tbl:logical-negation}
        $\begin{array}{ *{2}{c} }
            % 2 centered columns
            \toprule \\
            p & \neg q \\
            \midrule \\
            0 & 1 \\
            1 & 0 \\
            \bottomrule
        \end{array}$
    \end{subtable}%
    % Other Logical Connectives
    \begin{subtable}{0.5\linewidth}
        \centering
        \caption{Other Logical Connectives}
        \label{tbl:logic-connectives}
        $\begin{array}{ *{7}{c} }
            % 7 centered columns
            \toprule \\
            p & q & p \land q & p \lor q & p \lxor q & p \to q & p \iff q \\
            \midrule \\
            0 & 0 & 0 & 0 & 0 & 1 & 1 \\
            0 & 1 & 0 & 1 & 1 & 1 & 0 \\
            1 & 0 & 0 & 1 & 1 & 0 & 0 \\
            1 & 1 & 1 & 1 & 0 & 1 & 1 \\
            \bottomrule
        \end{array}$
    \end{subtable}
\end{table}

Note:
\begin{itemize}
    \item $p \land q$ is only true when p,q are true
    \item $p \lor q$ is only false when p,q are false
    \item $p \lxor q$ is only true when exactly one of p,q is true
    \item $p \implies q$ is true in all cases except when p is true and q is false
    \item $p \iff q$ is true only whe p,q have the same truth value
\end{itemize}

\subsection{Compound Statements}

\begin{table}[!htb]
    \caption{Truth Tables for $p \lor (q \land r)$ and $(p \lor q) \land r$}
    \centering
    % Order of Operations of Compound Statements
    \begin{subtable}{\linewidth}
        \centering
        \caption*{Order of Operations of Compound Statements}
        \label{tbl:logical-order-of-ops}
        $\begin{array}{ *{7}{c} }
            % 7 centered columns
            \toprule \\
            p & q & r & q \land r & p \lor (q \land r) & p \lor q & (p \lor q) \land r \\
            \midrule \\
            0 & 0 & 0 & 0 & 0 & 0 & 0 \\
            0 & 0 & 1 & 0 & 0 & 0 & 0 \\
            0 & 1 & 0 & 0 & 0 & 1 & 0 \\
            0 & 1 & 1 & 1 & 1 & 1 & 1 \\
            1 & 0 & 0 & 0 & 1 & 1 & 0 \\
            1 & 0 & 1 & 0 & 1 & 1 & 1 \\
            1 & 1 & 0 & 0 & 1 & 1 & 0 \\
            1 & 1 & 1 & 1 & 1 & 1 & 1 \\
            \bottomrule
        \end{array}$
    \end{subtable}%
\end{table}

As a result of this truth table, ambigious statements
such as $p \lor q \land r$ are to be avoided since their
meaning results in different propositions.

\begin{table}[!htb]
    \caption{Tautology \& Contradiction}
    \centering
    % Truth Tables for a Tautology \& Contradiction
    \begin{subtable}{\linewidth}
        \centering
        \caption*{Truth Tables for a Tautology \& Contradiction}
        \label{tbl:tautology-contradiction}
        $\begin{array}{ *{7}{c} }
            % 7 centered columns
            \toprule \\
            p & q & p \lor q & p \to (p \lor q) & \neg p & \neg p \land q & p \land (\neg p \land q) \\
            \midrule \\
            0 & 0 & 0 & 1 & 1 & 0 & 0 \\
            0 & 1 & 1 & 1 & 1 & 1 & 0 \\
            1 & 0 & 1 & 1 & 0 & 0 & 0 \\
            1 & 1 & 1 & 1 & 0 & 0 & 0 \\
            \bottomrule
        \end{array}$
    \end{subtable}%
\end{table}

\begin{definition}[Tautology]
A compound statement that is true for all truth value assignments for its individual component statements.
\end{definition}

\begin{definition}[Contradiction]
A compound statement that is false for all truth value assignements for its individual component statements.
\end{definition}

We will use the following symbols to denote these two terms:

% Define new terminology
\newcommand\taut{T_0}
\newcommand\contra{F_0}

\begin{tabularx}{0.80\linewidth}{@{}lX}
    & $\taut$: Tautology \\
    & $\contra$: Contradiction
\end{tabularx}

\subsection{Premises \& Arguments}

In general an argument begins with a list of given statements (premises),
and an implication known as the conclusion of the argument

\begin{definition}[Premise]
A premise is the given statement in an argument.
\end{definition}

Our process for arguing will be as follows:

\begin{enumerate}
    \item Examine the premises, $p_1, p_2, p_3, \ldots, p_n$
    \item Show that each of the premises is true, $p_1 \land p_2 \land p_3 \land \ldots \land p_n$ 
    \item Disprove or prove the implication.
\end{enumerate}

Where the implication is 
\begin{align*}
    (p_1 \land p_2 \land p_3 \land \ldots \land p_n) \to q
\end{align*}

and the hypothesis is the conjunction of the premises.

Note that:
\begin{itemize}
    \item If any of $p_n$ are false, then the implication is always true
    \item If every $p_n$ has the truth value 1, and q also has truth value 1,
        then the implication is a tautology and we have a valid argument.
\end{itemize}

\end{document}
