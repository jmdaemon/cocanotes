%\documentclass[../../../../main.tex]{subfiles}
%\graphicspath{{\subfix{../../../../figures/}, ./figures}}

%\begin{document}
\section{Logical Implication: Rules of Inference}
    
\begin{align*}
    (p_1 \land p_2 \land p_3 \land \cdots \land p_n) \to q
\end{align*}

$p_1, p_2$, and $p_3$ are the premises of the argument.

$q$ is the conclusion for the argument.

The preceding argument is \textit{valid} and the conclusion is true when all the premises $p_n$ are true.

If any of $p_n$ are false, then the hypothesis $(p_1 \land p_2 \land p_3 \land \cdots \land p_n)$ is false
and the implication $(p_1 \land p_2 \land p_3 \land \cdots \land p_n) \to q$ is automatically true
regardless of the truth value of q.

One method to establish validity of an argument is to show that 
the statement $(p_1 \land p_2 \land p_3 \land \cdots \land p_n) \to q$ is a tautology.

\begin{definition}[Logical Implication]
    Let $p,q$ be arbitrary statements. If $p \to q$ is a tautology, then
    p \textit{logically implies} q denoted $p \implies q$.
\end{definition}

% Table, Example 2.6
We found that for primitive statements $p, q$ $p \to (p \lor q)$ is a tautology.
Therefore $p \implies (p \lor q)$. As well because of the first substition rule,
$p \implies (p \lor q)$ holds for any statements $p, q$ whether primitive statements or not.

Let $p, q$ be arbitrary statements.

If $p \iff q$, then $p \leftrightarrow q$ is a tautology. Thus, $p \implies q$, and $q \implies p$.

$p \centernot\implies q$ indicates that $p \to q$ is not a tautology.

% Table 2.9, Example 2.8
We know for statements $p, q$,
\begin{align*}
    \neg (p \land q) \iff \neg p \lor \neg q
\end{align*}

%\end{document}
